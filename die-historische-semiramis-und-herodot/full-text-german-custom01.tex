\documentclass[a4paper, 11pt, oneside, polutonikogreek, german]{article}
\usepackage[utf8]{inputenc}
\usepackage[T1]{fontenc}
\usepackage{ebgaramond}
% Load encoding definitions (after font package)

\usepackage{textalpha}
\usepackage{bbding}
\usepackage{listings}
\lstset{basicstyle=\ttfamily}
\usepackage{wasysym}
\usepackage{cjhebrew}
\usepackage{semtrans}
\makeatletter
\renewcommand*\U[1]{\oalign{#1\crcr\hidewidth\ltx@sh@ft{-3ex}%
   \vbox to .2ex{\hbox{\u{}}\vss}\hidewidth}}
\makeatother

\usepackage[dvipsnames]{xcolor}
\usepackage{eso-pic,graphicx}
\usepackage[top=78mm, bottom=83mm, outer=98mm, inner=105mm, landscape]{geometry}
\setlength{\columnsep}{90pt}
\usepackage{etoolbox}    % for \patchcmd
\usepackage{sectsty}
\usepackage{tocloft}

\allsectionsfont{\bfseries\tiny}
\sectionfont{\bfseries\tiny}
\subsectionfont{\bfseries\tiny}
\subsubsectionfont{\bfseries\tiny}

% change color of text, example replace all \color{Goldenrod} with \color{lightgray}
\definecolor{myColor}{RGB}{0, 86, 63}

\makeatletter % change only the display of \thepage, but not \thepage itself:
\patchcmd{\ps@plain}{\thepage}{\bfseries\tiny\color{Black}{\thepage}}{}{}
\makeatother

\color{Black}

% Babel package:
\usepackage[german]{babel}

% With XeTeX$\$LuaTeX, load fontspec after babel to use Unicode
% fonts for Latin script and LGR for Greek:
\ifdefined\luatexversion \usepackage{fontspec}\fi
\ifdefined\XeTeXrevision \usepackage{fontspec}\fi

% "`Lipsiakos"' italic font `cbleipzig`:
\newcommand*{\lishape}{\fontencoding{LGR}\fontfamily{cmr}%
		 \fontshape{li}\selectfont}
\DeclareTextFontCommand{\textli}{\lishape}
\usepackage{booktabs}
\setlength{\emergencystretch}{15pt}
\usepackage{fancyhdr}
\usepackage{microtype}

\begin{document}
\bfseries
\AddToShipoutPictureBG{\includegraphics[width=\paperwidth,height=\paperheight]{assyria1.jpg}}
\renewcommand\thefootnote{\bfseries\tiny{\arabic{footnote}}}
\let\oldfootnote\footnote
    \renewcommand{\footnote}[1]{\oldfootnote{\tiny\bfseries#1}}
\begin{titlepage}
    \centering                     % Center everything on the page

    % -------------------------------------------------
    % Title rules
    % -------------------------------------------------
    \rule{\textwidth}{1.6pt}\\[-\baselineskip]\vskip2pt
    \rule{\textwidth}{0.4pt}\\[0.1\baselineskip]

    % -------------------------------------------------
    % Title
    % -------------------------------------------------
    {\scshape\large Die historische Semiramis und Herodot}\\[0.1\baselineskip]

    % -------------------------------------------------
    % Bottom title rules
    % -------------------------------------------------
    \rule{\textwidth}{0.4pt}\\[-\baselineskip]\vskip3.2pt
    \rule{\textwidth}{1.6pt}\\[0.1\baselineskip]

    % -------------------------------------------------
    % Autho
    % -------------------------------------------------
    
    {\scshape\small Von C. F. Lehmann}

    % -------------------------------------------------
    % Date / defense statement / author line
    % -------------------------------------------------

    \vfill
    % -------------------------------------------------
    % Place / publisher info (pushed to bottom)
    % -------------------------------------------------

    {\scshape\footnotesize \emph{Klio}.  Band 1}

    {\scshape\footnotesize Leipzig 1901}

    % -------------------------------------------------
    % Edition / license line
    % -------------------------------------------------
    {\scshape\scriptsize Solar Anamnesis Edition}\\[0.2\baselineskip]
    {\scshape\scriptsize CC0 1.0 Universell}
\end{titlepage}
\setlength{\parskip}{1mm plus1mm minus1mm}
\clearpage
\scriptsize
\paragraph{}
Dass der Name \emph{Sammuramat}, den eine, \emph{Adad-nirari\footnote{\emph{Adad-nirari}, nicht \emph{Rammân-nirari}. Dass der ideographisch geschriebene Name des assyrischen Wettergottes, in Personennamen dieser Art und überhaupt, \emph{Adad} und nicht \emph{Rammân} zu lesen ist, zeigt die phonetische Schreibung \emph{\textsuperscript{m}Ašur-ni-ra-ri-ni \textsuperscript{m}A-da-di-ni-ra-ri-e-\U{h}i} (K[önig]) (L[and]) \emph{Aššur-ni-i}: "`\emph{Ašur-nirari} (755-745 v. Chr. Vorgänger Tiglatpilesers 3.), Sohn \emph{Adad-niraris}, König von Assyrien"' in der von unserer deutschen Expedition nach Armenien in der Kirche Surb Po\r{g}os zu Van aufgefundenen Inschrift \emph{Sardurs} 3. von Urarṭu-Chaldia. Siehe \textsc{W. Belck} und \textsc{C. F. Lehmann}, \emph{Sitzungsberichte Berl. Ak., Phil hist. Kl.} 1899, 119 sub 6 (vgl. 1900, 624 sub 116) und \emph{Nachrichten der K. Gesellschaft d. Wissensch. zu Göttingen}, [GN.], Phil. hist. Kl. 1899, S. 83 f. sub 5. Schon vorher war die Lesung \emph{Adad} von \textsc{Oppert} (\emph{Zeitschrift für Assyriologie} 9. (1894) 310 und mir ib. 10. (1895) S. 87) gefordert worden.} 3.} von Assyrien (812-783) nahestehende königliche Frau trägt, mit dem der \emph{Semiramis} identisch ist, weiß man seit Langem,\footnote{Siehe \textsc{J. Oppert}, \emph{Histoire des Empires d'Assyrie et de Chaldée} (1865) p. 128-130. --- \textsc{G. Rawlinson}, \emph{The five great monarchies of the East}, vol. 2. p. 119-121. --- \textsc{Duncker}, \emph{Geschichte des Altertums} 2.\textsuperscript{5} S. 36, 254.} und ebenso alt sind die Versuche, diese \emph{Sammuramat} mit der \emph{Semiramis} der Sage, des ktesianischen Romans, in Verbindung zu bringen. Die "`ganz auffallende"' Angabe Herodots (1. 184), der ungefähr um die Zeit, da jene \emph{Sammuramat} wirklich lebte, eine \emph{Semiramis} kennt, hat, wie vor ihm \textsc{Oppert} und \textsc{G. Rawlinson} und nach ihm \textsc{Hommel},\footnote{\emph{Geschichte Babyloniens und Assyriens} (1885 ff.) S. 631 f.} \textsc{Ed. Meyer}\footnote{\emph{Geschichte des Altertums} [GA.] 1. (1884) § 411 Anm.} erneut hervorgehoben, dabei aber bestimmt in Abrede gestellt, dass die "`Semiramis des Ktesias"' irgendetwas mit der herodoteischen zu tun habe.

Ähnlich urteilten \textsc{Tiele}\footnote{\emph{Babylonisch-assyrische Geschichte} 1. (1886), 207; 212 f.} und \textsc{Winckler},\footnote{\emph{Geschichte Babyloniens und Assyriens} (1894), 120 f.} während ich\footnote{Rezension über \textsc{Wincklers} \emph{Geschichte Babyl. u. Assyr.; Berl. Philol. Wochenschrift} 1894, Sp. 239 f.} im Gegensatz zu des Letzteren Ausführungen die Forderung aussprach, dass unbedingt die Gestalt der historischen \emph{Sammuramat} als Kern und Ursprung der ktesianischen zu betrachten sei. Gleichzeitig wies ich andeutungsweise auf die Bedingungen und Umstände hin, die die Entstehung der Sage aus den über die wirkliche \emph{Semiramis} bekannten historischen Tatsachen durchaus ungezwungen erklären, was bisher von keinem der die Identität vertretenden Forscher ernstlich versucht war.\footnote{Insofern traf \textsc{Wilckes} Kritik, \emph{Hermes} 28. 187, zu.}

Skizzenhaft und nebensächlich wie sie geboten war, entging meine Erklärung der Beachtung.\footnote{Ausgenommen \textsc{J. Marquart}, \emph{Chronologische Untersuchungen}, S. 642[8] Anm. 21.} Und davon ganz abgesehen, tritt neuerdings die Neigung hervor, die Persönlichkeit auch der historischen \emph{Semiramis} möglichst zurückzudrängen. Die der Sagengestalt gegenüber erforderliche Skepsis wird unberechtigter Weise auf die geschichtliche Persönlichkeit übertragen, und so die historische Würdigung einer in mehr als einer Hinsicht interessanten und bedeutsamen Erscheinung erschwert und verhindert.

Während \textsc{Maspero}\footnote{\emph{Histoire ancienne des peuples de l'orient classique}, vol. 3. (1899), p. 98 u. n. 1.} sie unter Erwähnung der älteren Literatur seit \textsc{Oppert} möglichst kurz abtut und seinen Standpunkt als dem \textsc{Tieles} entsprechend kennzeichnet, wird sie in dem dem alten Westasien gewidmeten Teile von \textsc{Helmolts} \emph{Weltgeschichte}\footnote{\textsc{H. Winckler}, \emph{Das alte Westasien} in \textsc{Helmolts} \emph{Weltgeschichte} Bd. 3 (1899), S. 25 f.; S. 60 f.} überhaupt nicht erwähnt.\footnote{Sachgemäß dagegen \textsc{J. Krall}, \emph{Grundriss der altorientalischen Geschichte} 1. (1899), S. 137; "`... \emph{Sammuramat}, die Herrin, welche für eine babylonische Prinzessin und ein Vorbild der Semiramis der griechischen Sage gehalten wird."'} Und selbst auf dem Gebiet, auf dem die Bedeutung dieser Herrscherin relativ am wenigsten zu bestreiten sein wird, dem religionsgeschichtlichen, lässt sich derselbe Rückgang beobachten.

Es war \textsc{Tiele}, der in seiner \emph{babylonisch-assyrischen Geschichte} (a. a. O.) mit guten Gründen der \emph{Sammuramat} einen bedeutsamen Anteil an der Einführung des \emph{Nebo}-Kultes in Assyrien zuschrieb und hierbei auch in seiner Religionsgeschichte beharrte.\footnote{\textsc{C. P. Tiele}, \emph{Geschichte der Religion im Altertum bis auf Alexander des Großen.} Deutsch von \textsc{G. Gehrich}, 1. (1896), 191.} In \textsc{Jastrows}\footnote{\textsc{M. Jastrow} (\emph{Handbooks on the History of Religions}, vol. 2.): \emph{The religion of Babylonia and Assyria} (1898), cf. p. 128, 228. Vgl. über dieses Buch im Allgemeinen \textsc{Tieles} anerkennende Besprechung, \emph{Zeitschrift f. Assyriologie} [ZA.], 14., 184 ff.} neuester religionsgeschichtlicher Darstellung sieht man sich auch nur nach dem Namen der \emph{Sammuramat} vergebens um.

Bei dieser Sachlage erscheint es angezeigt, die Nachrichten über die historische Semiramis erneut zu prüfen und zu einem Gesamtbilde zu vereinigen, das sich von den Zügen der Sagengestalt deutlich abhebt. Dabei sind Wesen und Herkunft der herodoteischen Nachricht ins rechte Licht zu setzen. Dass hierauf und auf die Ergebnisse solcher Betrachtung für die Kritik des Herodot besonderes Gewicht gelegt wird, ist im Titel der vorliegenden Ausführungen angedeutet.

Erst wenn über die geschichtliche Herrscherin möglichste Klarheit gewonnen ist, darf der, unserer Überzeugung nach erfolgverheißende Versuch gemacht werden, sie als Ausgangspunkt und Grundlage für die Bildung der Sage zu erweisen.

Die Quellen für die historische \emph{Semiramis} sind:

1. Assyrische Inschrift\footnote{Veröffentlicht: Sir \textsc{H. Rawlinson}, \emph{The cuneiform inscriptions of Western Asia}, vol. 1. [I R.] (1861) 35 No. 2 und \emph{Keilschrifttexte zum Gebrauch bei Vorlesungen}, herausgegeben von \textsc{Ludwig Abel} und \textsc{Hugo Winckler}, S. 14; umschrieben und übersetzt zuletzt von \textsc{Abel} in Bd. 1. der \emph{Keilinschriftlichen Bibliothek} [KB. I], herausgegeben von \textsc{Eberhard Schrader}, S. 192/3.} auf mindestens zwei Statuen des Gottes \emph{Nabû}, (\emph{Nebo}) aus Kalach-Nimrud jetzt im Britischen Museum.\footnote{S. \emph{Guide to the Nimroud Central Saloon}, p. 8 No. 69 u. 70, Abbildung einer der Statuen s. z. \textsc{B. Hommel} a. a. O. S. 621.} Sie lautet,\footnote{Assyrisch: ¹ \emph{A-na} (G[ott]) \emph{Nabî da-pi-ni ša-ki-e abil E-sag-gil...} \textsuperscript{5} \emph{... na-ra-am Bêli bêl bêle,...} \textsuperscript{7} \emph{... a-šib E-zi-da ša \d{k}i-rib} (St[adt]) \emph{Kal-\U{h}i bêli rabî bêli-šu a-na balaṭ} \textsuperscript{m}(G.)\emph{Adad-nirari šar} (L.)\emph{Aššur bêli-šu u balaṭ} \textsuperscript{9} \textsuperscript{f}•\emph{Sa-am-mu-ra-mat aššat ekalli belti-šu \textsuperscript{m}Bêl-tar-ṣi-ili-ma} (M[ensch, Beamter])\emph{šakin} (St.)\emph{Kal-\U{h}i} (L.)\emph{\U{H}a-me-di} (L.)\emph{Sir-ga-na} (L.)\emph{Te-me-ni} (L.)\emph{Ia-lu-na} ¹¹ \emph{a-na balaṭ napšâti-šu arak ûme-šu šum-ud šanâti-šu šala(m)-mu bîti-šu u nisê-šu lâ bašî murṣi-šu} ¹² \emph{u-še-piš-ma a\d{k}îš. Ma-nu ar-ku-u a-na} (G.)\emph{Nabî na-at-kil ana ili ša-ni-ma la ta-tak-kil.} --- Zur Lesung und Deutung: \emph{arak} nicht \emph{arkat} (wie KB. 1. u. s. w.). Zu \emph{šum-ud} = \emph{šum'ud} ($\surd{}$\<m'd> Inf. 3. 1.) in KB. 1. nicht gedeutet, vgl. bes. VR. 34, Col. 3. 43 (\textsc{Zimmern}). --- Z. 12: \emph{a\d{k}îš. Ma-nu ar-ku-u}, nicht mit KB. I.: \emph{a\d{k}îš ma amelu ar-ku-u. Ma-nu} (für \emph{man-nu} "`\emph{quis, quisquis}"') \emph{ar-ku-u} "`Wer immer, (Du) Späterer"' (seiest); die Stelle fasst richtig auch \textsc{Delitzsch}, \emph{Assyrisches Handwörterbuch} [HW.] S. 419, nur ist in dem betreffenden Zitat \emph{man-nu} Druckfehler für \emph{ma-nu}.} möglichst wörtlich, übersetzt: "`Dem Gotte Nebo, dem Gewaltigen, Erhabenen, dem "`Sohne von Esaggil"' (folgen dessen gesamte Attribute), "`der da wohnt im Tempel \emph{Ezida} in Kalach hat dies (sc. die Statue), "`für das Leben"'\footnote{Ständige Formel bei derartigen Widmungen. Die Widmung erfolgt zumeist wie hier, zugleich "`für das Leben"' des Herrschers und "`für das Leben"' des weihenden Untergebenen. S. dazu \textsc{C. F. Lehmann}, \emph{Ein Siegelzylinder König Bur-Sin's von Isin, Beiträge zur Assyriologie und vergleichenden semitischen Sprachwissenschaft} (ed. \textsc{F. Delitzsch} u. \textsc{P. Haupt}) Bd. 2., S. 598 Anm. *** vgl. S. 621.} \emph{Adadniraris}, Königs von Assyrien und "`für das Leben"' der \emph{Sammuramat}, der Frau des Palastes, seiner Herrin, \emph{Bêl-tarṣi-iluma}, der Statthalter der Stadt Kalach, wie der Länder Chamidi, Sirgana, Temeni und Ialuna, (zugleich) "`für das Leben"' seiner Seele, die Länge seiner Tage, die Mehrung seiner Jahre, den Frieden seines Hauses und seiner Leute und auf dass ihn keine Krankheit betreffe, anfertigen lassen und geweiht. Wer immer Du später (sein mögest),\footnote{So reden auch die assyrischen und babylonischen Könige in den Fluch- und Segensformeln am Schlusse ihrer Inschriften ihre Nachfolger an. Vgl. z. B. die Steleninschrift \emph{Samaššumukîn's} (\textsc{C. F. Lehmann}, \emph{Šamaššumukîn, König von Babylonien u. s. w.} Teil 2., S. 10): \emph{Mannu ina šarrâni arkûtu}, "`Wer (immer) unter den späteren Königen,"' eine Wendung, die Herodot (1. 187) genau wiedergibt: τῶν τις ἐμεῦ ὕστερον γινομένων Βαβυλῶνος βασιλέων..., ein Beweis, dass Herodot sich in Babylonien Königsinschriften hat vorübersetzen lassen, wenn auch natürlich keine Inschrift des Weiteren so gelautet haben kann, wie Herodot fortfährt. Näheres s. u.} auf Nebo verlasse Dich, auf einen anderen Gott vertraue nicht."'

2. Herodot 1. 184 sq.: τῆς δὲ Βαβυλῶνος ταύτης πολλοὶ μέν κου καὶ ἄλλοι ἐγένοντο βασιλέες, τῶν ἐν τοῖσι Ασσυρίοισι λόγοισι μνήμην ποιήσομαι, οἳ τὰ τείχεά τε ἐπεκόσμησαν καὶ τὰ ἱρὰ, ἐν δὲ δὴ καὶ γυναῖκες δύο. ἡ μὲν πρότερον ἄρξασα, τῆς ὕστερον γενεῇσι πέντε πρότερον γενομένη, τᾗ οὔνομα ἦν Σεμίραμις, αὕτη μὲν ἀπεδεξατο χώματα ἀνὰ τὸ πεδίον ἐόντα ἀξιοθέητα · πρότερον δὲ ἐώθεε ὁ ποταμὸς ἀνὰ τὸ πεδίον πᾶν πελαγίζειν: ἡ δὲ δέυτερον γενομένη ταύτης βασίλεια, τῇ οὒνομα ἦν Νίτωκρις, αὕτη δὲ συνετωτέρη γενομένη τῆς πρότερον ἀρξάσης κτλ.

In Betracht kommt ferner

3. Josephus \emph{c. Ap.} (ed. \textsc{Niese}) 1. 142: Ταῦτα μὲν οὕτως ἱστόρηκεν (sc. ὁ Βηρῶσος § 130, 134) περὶ τοῦ προειρημένου βασιλέως (sc. Ναβοκοδροσόρου) καὶ πολλὰ πρὸς τούτοις ἐν τῇ τρίτῃ βίβλῳ τῶν Χαλδαϊκῶν, ἐν ἧ μέμφεται τοῖς Ἐλληνικοῖς συγγραφεῦσιν, ὡς μάτην οἰομένοις ὑπὸ Σεμιράμεως τῆς Ἀσσυρίας κτισθῆναι τὴν Βαβυλῶνα καὶ τὰ θαυμάσια κατασκευασθῆναι περὶ αὐτὴν ὑπ' ἐκείνης ἔργα ψευδῶς γεγραφόσι · καὶ κατὰ ταῦτα τὴν μὲν τῶν Χαλδαίων ἀναγραφὴν ἀξιόπιστον ἡγητέον · κτλ.

Die assyrische Inschrift gibt uns, wie \textsc{Tiele} a. a. O. erkannt hat, Kunde von der Einführung des Kultus des Gottes \emph{Nebo} in Assyrien. Der Vorgang ist genau auf das Jahr bestimmt. Die assyrische "`Eponymenliste mit Beischriften"' ("`Verwaltungsliste"')\footnote{Am bequemsten zugänglich: KB. 1., S. 288 ff.; 3., 2, S. 142 ff.} meldet zum Jahre 787: "`Nebo zieht in den neuen Tempel ein."'\footnote{\emph{(ilu) Nabû ana bîti ešši etarab.}} Die neuen Bilder\footnote{Hier wo es sich um Einführung eines für Assyrien neuen Kults handelte, hat man sich offenbar mit der Anfertigung neuer Bilder des Gottes, natürlich in möglichst engem Anschluss an das Kultbild in Borsippa, begnügt, während in Fällen, wo eine alte Götterstatue abhandengekommen war, große Anstrengungen gemacht wurden, die Kontinuität zwischen dem neu hergestellten und dem alten Kultbilde scheinbar zu wahren. So wird, als \emph{Nabûbaliddin} von Babylonien (um 870) das vorzeiten von den räuberischen Sutäern vernichtete Bild des Sonnengottes (\emph{Šamaš}) von Sippar herstellen will, der auf Ton gezeichnete Entwurf des alten Bildes von einem Priester des Gottes im Euphrat "`gefunden"' und vorgewiesen --- ein frommer Betrug, von dessen Zustandekommen die Inschrift des Königs (s. KB. 3., 1 S. 174 ff.) der Mit- und Nachwelt ausführliche Kunde gibt. --- Außer den Statuen mit der Inschrift wurden noch unbeschriebene größere Statuen im \emph{Nebo}-Tempel zu Kalach aufgestellt. Die Angaben über die Zahl der gefundenen Statuen schwanken. Z. B. spricht Sir \textsc{H. Rawlinson}, 1. R. 35 No. 2 von, im Ganzen, fünf lebensgroßen beschriebenen (wovon zwei im Britisch Museum) und zwei kolossalen unbeschriebenen Statuen. \textsc{George Rawlinson} (a. a. O.) dagegen weiß von vier beschriebenen und vier inschriftlosen kolossalen Statuen zu berichten, die sein Bruder Sir \textsc{Henry} gefunden habe. Was nicht ins Britische Museum verbracht wurde, ward von Einheimischen verschleppt, oder blieb unbedeckt an Ort und Stelle stehen und liegen. An der Stätte des alten \emph{Nebo}-Tempels von Kalach, im Südosten der Ruinen ragt noch heute eine Statue halb aus der Erde hervor, deren bei meinem Besuche von Nimrud am 1./4. 1899 von mir genommene Photographie sie als ein Duplikat der in London befindlichen \emph{Nebo}-Statuen erweist. Ich benutze die Gelegenheit, um, wie in meinem Vortrage \emph{Armenien und Nordmesopotamien in Altertum und Gegenwart} darauf hinzuweisen, dass aus den in Nimrud herrenlos umherliegenden und der langsamen Vernichtung durch die Witterung preisgegebenen Stierkolossen, Statuen und Inschriftenplatten manches Museum sich mit verhältnismäßig geringer Mühe einen wertvollen Bestand an Denkmälern assyrischer Kultur verschaffen könnte.} des Gottes sind ohne Zweifel gleich zu Anfang bei Eröffnung des Tempels (im Südosten der Stadt Kalach\footnote{\emph{Kaha\U{h}-Nimrud} mit der Erdpyramide, unter der sich der Stufenturm des \emph{Ninib}-Tempels verbirgt, präsentiert sich bekanntlich heute noch ähnlich wie zu Xenophons Zeiten. Auch die (von Xenophon 3. 4, 7 u. 10) geschilderte Struktur der κρηπίς, besonders da wo die Mauern dem damals unmittelbar an der Stadt vorbeifließenden, heute um 1-2 Kilometer entfernten Tigris ausgesetzt waren, lässt sich in Nimrud gut beobachten. Xenophons bisher gänzlich unaufgeklärte Bezeichnungen Λάρισσα für Kalach und Μέσπιλα für Niniveh betrachte ich (\emph{Zeitschr. f. Assyr.} 14., 122 Anm. 3) "`als Missverständnisse aus dem Aramäischen, \<l:re+sA'> (\emph{rêša} Kopf, Oberteil) \<ma+sp:lA'>, \<mE+s:p:lA'> oder eine ähnliche Ableitung von \<+spl> "`niedrig sein."' \textsc{Nöldeke} stimmt mir, was die Wortform anlangt, zu: "`ersteres mag damals noch \emph{lārēšā}, letzteres \emph{mešpilā}"' gelautet haben. In welchem Sinne freilich diese Bezeichnungen von den aramäisch redenden Führern der 10000 auf die beiden Ruinenstätten angewendet sein mögen, bleibt aufzuklären."'} aufgestellt worden. Der Kult hat tatsächlich Eingang und Verbreitung gefunden. Wie \textsc{Tiele} betont, begegnen wir vor dem Jahre 787 keinen mit \emph{Nebo} zusammengesetzten assyrischen Namen. Gleich die Eponymen (\emph{limu}) je des Jahres 785 und 777 heißen dagegen \emph{Nabû-šar-uṣur} und \emph{Nabû-pur-ukîn}.

\emph{Nabû} (\emph{Nebo}) ist ein babylonischer Gott. Hauptsächlich wurde er in \emph{Borsippa} verehrt, das rechts des Euphrat, in Babylons nächster Nähe belegen war. Der Tempel in \emph{Borsippa} hieß \emph{Ezida}. Der Name ist sumerisch und wird von den Babyloniern selbst durch \emph{bîtu kînu}, dass "`beständige, legitime Haus"' wiedergeben. Der zugehörige Stufenturm war \emph{E-ur-imin-an-ki}\footnote{S. die auf die Herstellung dieses Stufenturms bezügliche "`Borsippa-Inschrift"' Nebukadnezars 2. (vgl. KB. 3., 2, S. 52 ff.) Col. 1. 27.} ("`Tempel der sieben Abteilungen Himmels und der Erde"') benannt. \emph{Nebo} der Weise, Allwissende,\footnote{\emph{dupšar gimri} z. B. in \emph{Sargon's 2.} Zylinder-Inschrift Z. 59 (\textsc{Lyon}, \emph{Keilschrifttexte Sargon's} S. 36), in \emph{Asurbanabal's großer Thontafelinchrift L} \textsuperscript{4}, Col. 1., 11 (\textsc{C. F. Lehmann}, \emph{Šamaššumukîn} Teil 2. S. 22).} der "`Tafelschreiber des Alls,"'\footnote{Z. 22 der an Nebo gerichteten Inschrift auf einer in Babylon gefundenen kleinen Stele \emph{Šamaššumukîn's} (\textsc{Lehmann} a. a. O. l. c., Titelblatt und Teil 2. S. 10.)} sowie des, ein Abbild des Weltalls nach babylonischer Anschauung darstellenden Tempels \emph{Esaggil},\footnote{\textsc{Jensen}, Kosmologie S. 234 ff. Vgl. zuletzt \textsc{Hommel}, \emph{Das babylonische Weltbild, Aufsätze und Abhandlungen} 3. S. 344 ff.} zugleich und ursprünglich der Beschützer der Kenntnisse und Fertigkeiten, die den Ackerbau fördern und bedingen,\footnote{\textsc{Jastrow} a. a. O., p. 125.} galt in der offiziellen babylonischen Theologie der späteren Zeit als der Sohn des \emph{Marduk}, des Hauptgottes von Babylon. Deshalb befand sich auch in dem dem \emph{Bêl-Marduk} geweihten Haupttempel von Babylon, \emph{Esaggil}, (zu welchem der Stufenturm \emph{E-temen-an-ki}, "`Tempel der Grundfesten des Himmels und der Erde,"' gehörte), wie für den Vater des \emph{Marduk, Ea} (\emph{Aë}), den "`König des Ozeans"' (babyl. \emph{šar apsî}\footnote{Über die Herkunft der ersten babylonischen Dynastie und der Hauptkulte der Stadt Babylon (und damit Babyloniens überhaupt), siehe vorderhand meine Bemerkungen, \emph{Zwei Hauptprobleme} S. 162 f. Anm. 3, 214 f. (\emph{Aë}) \emph{šar apsî} ist der für Alexander den Großen in seiner letzten Krankheit, nach dem natürlich nicht zu bezweifelnden Zeugnis der Ephemeriden, in Babylon befragte Heilsgott Σάραπις. Es ist dieser Gott, den Ptolemaios 1. eingeführt und mit dem ägyptischen \emph{Osorapis} identifiziert hat. Bei der Einführung spielen, wie bei so vielen Maßnahmen der ersten Ptolemäer der Wetteifer mit den Seleukiden und die beiderseitigen Ansprüche auf die Weltherrschaft eine Rolle. Vgl. einstweilen meine Ausführungen Verhandl. \emph{Berl. Archäol. Ges.} November 1897 = \emph{Wochenschr. f. klass. Philologie} 1898, No. 1 Sp. 26 ff. --- Ausführlicheres darüber demnächst. [Die Tradition der Ephemeriden Alexanders des Großen habe ich, \emph{Hermes} 36, 319 f. zurückzuführen gesucht auf das Hauptexemplar, das Ptolemaios nach Perdikkas Ermordung an sich genommen und auf eine in Eumenes' Händen verbliebene Kopie (resp. sein Konzept), in die Hieronymos von Kardia Einsicht erhielt. Da Ptolemaios nach Perdikkas' Tode als Sieger aufzutreten vermied, waren die von mir dabei gebrauchten Ausdrücke "`Besiegung"' und "`Beute"' unrichtig gewählt.]}), so für \emph{Marduk's} Sohn \emph{Nebo} ein Heiligtum, das ebenfalls den Namen \emph{Ezida} führte, vielfach mit der unterscheidenden Bezeichnung: "`welches in \emph{Esaggil} (belegen ist)."'

Die Einführung eines babylonischen Kultes in Assyrien setzt natürlich besonders enge und, mindestens dem äußeren Scheine nach, friedliche Beziehungen zwischen Assyrien und Babylonien voraus, die in dieser Periode nur denkbar sind auf der Grundlage assyrischer Syrematie. Solche Beziehungen lassen denn auch die vorhandenen Nachrichten deutlich genug erkennen.

\emph{Adadmirari's 3.} Vater, \emph{Šamši-Adad}\footnote{Gewöhnlich \emph{Šamši-Rammân} genannt, vgl. S. 256 Anm. 1, er ist der vierte assyrische Fürst dieses Namens, der zweite, der den Königstitel trägt, vgl. \emph{Zwei Hauptprobleme} Tab. 4. Anm. 10, Tab. 5. Anm. 13.} (825-812), Sohn \emph{Salmanassar's 2.} (860-825) hatte mehrfache Kämpfe mit Babylonien zu bestehen. Gegen den Anfang seiner Regierung besiegte er auf seinem, nach seiner Schilderung\footnote{Inschrift des Königs, Col. 4., s. KB. 1. S. 184. In Z. 24 beruht die nach \textsc{Hommel} eingeführte Ergänzung \emph{Bau-a\U{h}-iddin}, wie längst erkannt, auf einem Versehen. In Zeile 37 wird ja im Bericht über eben diesen Feldzug noch \emph{Mardukbalâ(t)su-i\d{k}bî} der Vorgänger \emph{Bau-a\U{h}-iddin's} genannt. Die vier Feldzüge von denen der Text allein berichtet, fallen sicher sämtlich in den ersten Teil dieser Regierung, vor 817, und zwar in die Lücke, die in der Verwaltungsliste zwischen den Jahren 822 und 817 klafft. Die Jahre 825-22 (s. KB. 3. 1, S. 144) füllt der bereits unter Salmanassar 2. ausgebrochene Aufstand aus. Die drei ersten Züge der Inschrift sind gegen \emph{Naïri} gerichtet, die Verwaltungsliste weist von 817 ab keinen Zug gegen \emph{Naïri} auf. Die drei Feldzüge gegen \emph{Naïri} stellen zudem offenbar nur ein, auf drei aufeinanderfolgende Kriegsjahre (821-18) verteiltes Unternehmen dar. Dem Zuge gegen Chaldäa, den die Liste für 813 verzeichnet, gehen voraus die gegen andere Landschaften gerichteten Züge der Jahre 817, 816, 814. (Für 815 ist kein Feldzug verzeichnet.) Folglich kann der gegen \emph{Mardukbala(ṭ)suikbî} und Chaldäa gerichtete vierte Feldzug nicht der vom Jahre 813 sein, sondern fällt in das vor 817 allein freibleibende Jahr 818 (nicht 821, wie irrig \emph{Zwei Hauptprobleme} S. 47).} an Erfolgen reichen, vierten Feldzug den babylonischen König \emph{Marduk-bala(ṭ)su-i\d{k}bî} in entscheidender Schlacht. Nachdem aber in Babylonien ein Regierungswechsel stattgefunden hatte, begann der Krieg aufs neue und endete mit der Gefangennahme des babylonischen Königs \emph{Bau-a\U{h}-iddin}, den Jener "`samt seiner Habe und dem Schatz seines Palastes nach Assur"' verbringen ließ. "`\emph{Samsi-Adad} stieg bis nach Chaldäa,"'\footnote{Die Bedeutung der Chaldäer als eines von den Babyloniern wohl zu unterscheidenden Volkes ist von \textsc{Délattre} und \textsc{Winckler} klargestellt worden, s. \textsc{Winckler}, \emph{Untersuchungen zur altorientalischen Geschichte}, S. 47 ff. und sonst vielfach. \emph{Nabopolassar} und seine Dynastie sind Chaldäer. --- Über den Ursprung der Bezeichnung Χαλδαῖοι für die babylonische Priesterschaft oder gewisse Klassen derselben s. \textsc{Lehmann}, \emph{Šamaššumukîn}, Teil 1., S. 173.} d. h. dem Süden des babylonischen Tieflandes hinab, empfing den Tribut der "`Könige"' des Chaldäerlandes und diktierte dann den Babyloniern den Frieden, bei dem wie gewöhnlich eine Grenzregulierung eine Hauptrolle spielte.\footnote{Synchronistische Geschichte, Col. 4., 1 ff. --- Der Name des Assyrerkönigs ist weggebrochen. Den Abschnitt \emph{Adadnirari 3.} zuzuschreiben, wie \textsc{Winckler}, \emph{Untersuchungen} S. 25, \emph{Geschichte} S. 117 ff., 203, \emph{Ein Beitrag zur Geschichte der Assyriologie in Deutschland} S. 24 ff., 41 f. und jetzt, ihm folgend, \textsc{Maspero}, \emph{Histoire} 3., 98 n. 3, wollen, ist deshalb unmöglich, weil, wie \textsc{Wilcken}, \emph{Zeitschrift der Deutschen Morgenländischen Gesellschaft} [ZDMG.] 47 (1893) S. 481 u. 712 Anm. 1, richtig hervorhebt, die diesen Abschnitt abschließende Grenzregulierung in dem ganzen Dokument das scheidende Moment abgibt: auf die Grenzregulierung folgt immer ein neuer Assyrerkönig, wie ja auch (s. im Text) der mit \emph{Adadnirari's} Namen beginnende Abschnitt in die Grenzregulierung ausläuft. In derselben Weise wird in dem babylonischen Paralleldokument, der "`Chronik P.,"' die Trennung der babylonischen Könige gehandhabt. Gegen die Versuche sich auch hier über die Trennungslinie hinwegzusetzen, s. meine Ausführungen \emph{Zwei Hauptprobleme}, S. 66, die ich auch jetzt aufrechterhalte; vgl. dazu \textsc{Jensen}, GGN., 1900 S. 881.} Dies geschah im Jahre 813 (dem letzten vollen Regierungsjahr \emph{Šamši-Adad's}), zu welchem die Verwaltungsliste den Vermerk "`nach Chaldäa"' bietet.

\emph{Adad-nirari 3.}, sein Sohn, "`führte"' nach der "`synchronistischen Geschichte"' Col. 4., 14 ff. "`die geraubten Leute zurück."'\footnote{\emph{Niše šallûti ana ašrišu ut[era].} Die darauffolgenden Worte \emph{iš\d{k}u} (s. \textsc{Delitzsch}, HW. S. 147b) \emph{ginâ} ŠE. PATpl. \emph{ukinšunuti} bedeuten vielleicht: "`und setzte sie in ihren legitimen(\emph{ginû}) Besitz... wieder ein."'} Von einem Kriege mit Babylonien ist nicht die Rede: der Sohn heilt z. T. die Wunden, die der Vater hat schlagen müssen. Ebenso wenig ist in den eigenen Inschriften \emph{Adadnirari's} von kriegerischen Unternehmungen gegen Babylonien die Rede. Im Gegenteil: "`die Städte Babylon, Borsippa, Kutha überbrachten den Ruf der Götter \emph{Bêl(-Marduk)}, \emph{Nabû} und \emph{Nergal}."'\footnote{\emph{Babilu, Barsip Kutû ri\U{h}at} (HW. 616 a) \emph{Bêli, Nabî, Nergal lû iššûni.}} Die dann folgende Erwähnung von Opfern\footnote{\emph{nî\d{k}e ellûte} = "`reine Opfer."'} unmittelbar vor dem Bruch, dem der Schluss der betreffenden gleichfalls aus Kalach stammenden Inschrift zum Opfer gefallen ist, wird man nach Analogie entsprechender Nachrichten über andere Könige, so besonders über \emph{Adadnirari's 3.} Großvater \emph{Salmanassar 2.} dahin ergänzen dürfen, dass der König selbst diese Opfer an Ort und Stelle dargebracht habe. Für diesen Vorgang passt in der für \emph{Adadnirari 3.} und die folgenden Könige vollständig vorliegenden Verwaltungsliste nur das Jahr 812, in welchem \emph{Šamši-Adad} starb und \emph{Adadnirari 3.} die Regierung antrat. Der König, der nach dieser Angabe "`nach Babylon"' zog, wird \emph{Adadnirari}, nicht etwa noch sein Vater gewesen sein. Die Zurückführung der gefangenen Babylonier und die Vornahme der Opfer in den babylonischen Städten, zum Ausdruck und in Ausübung der Oberhoheit über Assyrien, wären des Königs erste Regierungshandlungen gewesen. Wie sich dazu fügt, was wir über \emph{Sammuramat} erfahren und zu erschließen haben, werden wir alsbald sehen. Einstweilen handelt es sich vom Standpunkt dieser Untersuchung aus darum, möglichst tief in das historische Verständnis der Sachlage einzudringen, ohne einen etwaigen Einfluss der \emph{Semiramis} auf die Gestaltung der Dinge zu berücksichtigen.

Abgesehen von einem Zuge gegen einen offenbar in dieser Periode besonders unruhigen und aufsässigen Aramäerstaat und vielleicht einem Zuge nach der Meeresküste\footnote{Gegen die \emph{Itu'(a)} (791). Auch unter den folgenden Assyrerkönigen werden mehrfach Züge gegen sie nötig, so 783, 777, 769 v. Chr. Ob im Jahre 803 überhaupt von einem Kriegszuge die Rede ist, bleibt zweifelhaft. Die Worte \emph{ana eli tamdi mûtânu} ("`nach der Seeküste. Pest,"' so KB. 1. 207), können ebenso wohl besagen, "`an der Meeresküste (herrschte die) Pest."'} hat nach der Verwaltungsliste während der ganzen langen, an Kriegen und militärischen Erfolgen reichen Regierung \emph{Adadnirari's 3.} keine Feindseligkeit auf babylonischem Boden stattgefunden. Die Erklärung wird in der mit Sicherheit erkennbaren Tatsache liegen, dass \emph{Adadnirari 3.} selbst die Zügel in der Hand behielt, keinen wie immer schattenhaften "`König von Babylon(ien)"' neben und unter sich duldete. Dafür spricht zunächst im Allgemeinen der Schluss der "`synchronistischen Geschichte."' Dieses Dokument stellt in Wahrheit einen Auszug aus den Archiven dar, der unter \emph{Adadnirari 3.} wegen der zwischen Babylonien und Assyrien streitigen Gebiete resp. speziell eines solchen Gebietes gefertigt wurde (KB. 1., 194 Anm. 1). Es schließt mit einer Mahnung an etwaige spätere babylonische Könige den Inhalt der Tafel dauernd zu beherzigen. Dabei werden die Überlegenheit Assurs und die Frevel Babyloniens aufs Nachdrücklichste betont. Unter \emph{Adadnirari 3.} haben also diese Grenzstreitigkeiten ihre definitive Erledigung im Assyrien günstigen Sinne gefunden.

Diesem Schlusspassus voraus geht der Bericht über die Grenzregelung (s. o.). Und hier findet sich ein Satz, der mit E. nicht anders gedeutet werden kann, als dass die Bewohner von Assyrien und Babylonien bei dieser Ordnung der Verhältnisse beteiligt sind.\footnote{\emph{Zwei Hauptprobleme}, S. 47 Anm. 3. Dass es sich wirklich um eine Beteiligung der Völker bei der Grenzregulierung handelt, zeigt unser Passus: \emph{nîše (mât) Aššur (mât) Karduniaš itti a\U{h}amiš [ib-ba-.], miṣru ta\U{h}umu ište(n)-niš} (nicht \emph{kêniš}) \emph{u-ki[n-nu]} (Plural, Ergänzung mit KB. 1. a. a. O. durch den Raum gefordert); "`Die Leute von Assyrien und Babylonien verständigten sich (?) (\emph{ib-ba-a: ibbâ} von \emph{nabû}, \textsc{Zimmern}) oder kamen zusammen (\emph{ib-ba-'-a, ibbâ'a} von \emph{bâ'u} kommen [?] vgl. \textsc{Meissner}, \emph{Suppl.} S. 21), Gebiet und Grenze bestimmten sie."' Vorher bewegt sich der Bericht auch nach den bisherigen Lesungen (s. o. S. 262 u. Anm. 3) im Singular, also können nur die "`Assyrer und Babylonier"' diejenigen sein, die gemeinsam die Grenze festsetzen. An den beiden anderen Stellen Col. 2. 37, 3. 19 könnte man zur Not, unter Annahme eines Subjektwechsels, die Sache so deuten, dass die Bewohner der beiden Länder zum friedlichen Verkehr angehalten worden wären, dass aber die Grenzregulierung nur von den vorher genannten Königen der beiden Länder ausgegangen wäre. Übrigens waren auch Assyrien und Babylonien, so wenig wie Makedonien und die makedonischen Reiche, reine Despotieen: um seinem Sohn \emph{Asurbanabal} als Thronfolger die Anerkennung zu sichern, beruft z. B. \emph{Asarhaddon} die "`Bewohner von Assur, die Großen und die Kleinen, vom oberen und vom unteren Meere"' (s. \textsc{C. F. Lehmann}, \emph{Šamaššumukîn} T. 1. 34 f., 2. 25).} Einer entsprechenden Wendung begegnen wir in den erhaltenen Teilen des Dokuments nur noch zweimal: unter \emph{Asurbêlkala} (dem Enkel des um 1010 v. Chr.\footnote{\emph{Zwei Hauptprobleme} S. 99 und passim. S. dazu \textsc{Ed. Meyer}, \emph{Lit. Centralbl.} 1899, Sp. 119 ff.; \textsc{Prašek}, \emph{Berl. Phil. Wochenschr.} 1898, Sp. 1296 ff.; \textsc{Tiele}, ZA. 14., 390 ff. Nach der bisher herrschenden und jetzt noch mehrfach vertretenen Meinung (z. B. \textsc{Jensen}, GGN., 1900, Nr. 11, 12) ein Jahrhundert früher.} herrschenden \emph{Tiglatpileser 1.}), der die Tochter des in Babylonien damals herrschenden \emph{Usurpator's} geheiratet hatte und ferner unter \emph{Adadnirari 2.} (911(?)-891), der den babylonischen König \emph{Nabûšumiškum} als Gefangenen nach Assyrien verbracht hatte, um dann mit ihm einen, gleichfalls von einer Verschwägerung\footnote{Synchron. Gesch. Col. 3. 17.} und einer Grenzregelung begleiteten Frieden zu schließen.

In beiden Fällen war also die babylonische Regierungsgewalt durch das Überwiegen assyrischen Einflusses beschränkt und so gut wie ausgeschaltet.

Bei \emph{Adadnirari 3.} aber tritt dessen alleiniges Auftreten sowohl wie die Gemeinsamkeit der beiden Völker so besonders nachdrücklich hervor,\footnote{\emph{istenis}, gemeinsam vgl. Anm. 1.} dass es, wie daraufhin schon früher von mir geschehen,\footnote{Dass \emph{Adadnirari 3.} selbst wahrscheinlich als Nachfolger \emph{Bau-ah-iddin's} in Dynastie H. der Königsliste figuriert habe, wurde von mir, \emph{Zwei Hauptprobleme}, S. 128 Anm. 3, ausdrücklich betont. Dies ist von \textsc{Hommel}, in seiner neuen, mir gerade noch benutzbaren Schrift "`\emph{Ein neuer babylonischer König}"' (Sitzungsber. der Königl. Böhm. Ges. d. W., Klasse für Philosophie, Geschichte und Philologie 1901, Nr. 5.), S. 10, der in der Einfügung des \emph{Adad-nirari 3.} resp. der \emph{Sammuramat} einen Fortschritt gegen meine Aufstellungen erblickt, übersehen worden.} als wahrscheinlich bezeichnet werden muss, er selbst sei als Nachfolger \emph{Ban-a\U{h}-iddin's} auf dem babylonischen Throne betrachtet worden.

Die Beweiskette lässt sich aber vollständig schließen indem ein Bedenken wegfällt, das sich aus dem bisher vorliegenden Wortlaut des zum Teil traurig verstümmelten, \emph{Adadnirari 3.} betreffenden Passus der synchronistischen Geschichte ergab.

Nach \textsc{Wincklers} Ausgabe\footnote{\emph{Untersuchungen zur altorientalischen Geschichte}, (1889) S. 151.} folgt in Col. 4. Z. 14 auf den Namen und des Titel \emph{Adadnirari's} unmittelbar vor dem abgebrochenen Ende der Zeile ein "`Winkelhaken,"' der als Zeichen für \emph{u} "`und"' (resp. als Anfangskeil des dafür verwendeten komplexen Zeichens) zu betrachten nahelag.\footnote{So auch KB. 1. S. 203.} Danach bestand trotz Allem die Möglichkeit, ja eine gewisse Wahrscheinlichkeit, dass, wie in den übrigen Abschnitten dieses Dokumentes, neben dem Assyrer ein babylonischer König genannt war.\footnote{Mit \emph{u} sind die Namen verbunden, z. B. Col. 1. 2, 5, asyndetisch stehen sie Col. 1. 24, 2. 14.} Für den Titel "`König von \emph{Karduniaš}"' wäre freilich hinter dem zu ergänzenden Eigennamen kein Raum verblieben. Aber man hatte zunächst an \emph{Bau-a\U{h}-iddin} zu denken, der schon früher genannt war. In solchem Falle ist die allerdings mehrfach zu beobachtende Wiederholung der Titulatur\footnote{Z. B.: 1. 18 vgl. mit 19; 2. 25 mit 29, 33; 3. 1 mit 10; 3. 23 mit 26.} in der "`synchronistischen Geschichte"' nicht unerlässlich.\footnote{Z. B.: 1. 8 vgl. mit 10, 13; 1. 24 mit 26; 2. (Haupttafel) 4 mit 9 (vgl. 2. 6, 8).}

Außerdem aber blieb andererseits der Verdacht lebendig, dass das betreffende, dem Bruch vorausgehende Häkchen um ein Unmerkliches anders gestaltet oder gestellt und der Rest des weiblichen Personendeterminativs sein könne, hinter welchem unweigerlich \emph{Sammuramat} zu ergänzen gewesen wäre.

Eine erneute Kollation der Stelle war somit dringend geboten. Ich wandte mich dieserhalb, ohne nähere Angabe meiner Zweifel, an Herrn \textsc{L. W. King} vom Britischen Museum, dem ich für die prompte Erledigung meines Gesuches aufrichtig dankbar bin. Seine Kopie zeigt --- und \textsc{King} fügt es nochmals disertis verbis hinzu --- dass das auf \emph{Adadnirari šarri mâti Aššur} folgende Zeichen nicht ein Winkelhaken sondern ein freistehender senkrechter Keil ist, auf den vor dem abgebrochenen Zeilenende noch Reste zweier weiterer Zeichen folgen.

Die einzige Lesung und Ergänzung, die danach noch ernstlich in Betracht kommen kann,\footnote{Der senkrechte Keil, auf den nach \textsc{Kings} Kollation deutliche Spuren des Zeichens \emph{ilu} "`Gott"' folgen, bezeichnet entweder die Präposition \emph{ana} oder ist Determinativ vor männlichen Personennamen. Die Entscheidung bringt das Verbum in Z. 15. So lange man den Satz wegen des vermeintlichen "`und"' auf zwei Personen zu deuten hatte, musste auf eine Pluralformel geschlossen und \emph{ik-nu-n[i] i\d{k}nûni} ergänzt werden (so KB. 1.), womit freilich für die Deutung nichts anzufangen war. Aus letzterem Grunde ist auch die Ergänzung \emph{\textsuperscript{m}(ilu) B[a-u-a\U{h}-iddin]} ausgeschlossen, ganz davon abgesehen, dass die Spuren des letzten erhaltenen Zeichens, wie \textsc{King} sie gibt, nicht wohl zu \emph{ba} passen würden. Der Senkrechte ist also nicht Personendeterminativ. Zu lesen und zu ergänzen ist vielmehr --- unter Ausnutzung der Spuren und z. T. unter Verwertung von Vorschlägen \textsc{H. Zimmerns} --- ¹\textsuperscript{4} \emph{\textsuperscript{m}(ilu) Adadnirari šar (mât) Aššur ana ili Na[bî u Marduk?]} ¹\textsuperscript{5} \emph{ik-nu-uš.} "`\emph{Adadnirari}, König von Assyrien beugte sich (\emph{iknuš}) vor (dem Gotte) \emph{Nebo} [und \emph{Marduk}(?)]."' Durch diesen Sachverhalt werden \textsc{Hommels}, ohne Kenntnis dieser neuen Kollation gezogene scharfsinnige Schlüsse in wesentlichen Teilen bestätigt. (A. a. O. S. 11: man vgl. \emph{ik-nu-uš}(?)"' [\emph{ana Marduk}?...]). In Zeile 15 a. E. mit KB. 1. vor \emph{ma-du-[ti]} "`viele,"' \emph{[dik]-tu} "`Kampf"' zu ergänzen, erschien von vornherein unwahrscheinlich, jetzt sprechen auch die Spuren bei \textsc{King} dagegen.} besagt, dass \emph{Adadnirari 3.} sich vor dem Gotte "`\emph{Na[bû} (und \emph{Marduk}?)] gebeugt"' hat, und dies scheint (nach Zeile 16) "`unter Freude und Jubel"'\footnote{\emph{i-na \U{h}i-[da-a-ti u ri-ša(-a)]-t[i].} Vor \emph{t[i]} Spuren, zu \emph{a} oder \emph{ša} ergänzbar. \emph{Ina (\U{h}idâti u) rišâti}, ständige Formel bei kultisch erfreulichen Ereignissen. Die folgende Zeile (17) endet bei \textsc{King} \emph{... ma-ni u ilâni} ("`Götter"'). Merkwürdig ist, dass in diesem Abschnitt \emph{ana} ideographisch, \emph{ina} phonetisch geschrieben erscheint, während es in dem ganzen übrigen Dokument umgekehrt ist, eine Einheitlichkeit die schwerlich den Archivauszügen selbst, eher einem (möglicher Weise von dem Verfasser des, \emph{Adadnirari 3.} betreffenden, Schlusspassus verschiedenen) Redaktor derselben zuzuschreiben sein wird.} geschehen zu sein. Ebenso wie die Inschrift \emph{Adadnirari's 3.} aus Kalach (oben S. 263) berichtet also die "`synchronistische Geschichte"' von der den babylonischen Staatsgöttern dargebrachten Verehrung des Assyrerkönigs, und zwar in engem Zusammenhang mit der Zurückführung der Gefangenen (Col. 4., 18), die gleich am Anfange seiner Regierung erfolgt sein muss. Wie der Bau des Palastes in Kalac\U{h}, dem jene Steinplatte mit der Inschrift entstammt, dem Bau des Nebotempels sicher voraussetzt, so wird auch die "`synchronistische Geschichte"' vor Einführung des Nebokultes in Assyrien abgefasst sein.

Von hier aus fällt denn auch auf diesen Vorgang ein erklärendes Schlaglicht.

Die Einführung des Nebokultes lieferte die Möglichkeit, das babylonische Königtum auf assyrischem Boden rite zu erwerben oder vielmehr ein Surrogat für diese Möglichkeit. Als "`König von Babylon,"' d. h. als legitimer König von Babylonien galt nur, wer am Jahresanfang in \emph{Esaggil} die Hände \emph{Bêl-Marduk}'s erfasst hatte,\footnote{Vgl. auch diese \emph{Beiträge}, S. 32.} und diese Zeremonie musste, wie es scheint, an jedem Neujahrstage wiederholt werden. Die Assyrerkönige, die faktisch über Babylonien herrschten, haben sich in späterer Zeit, diesem Brauche unterzogen: sie haben das babylonische Königtum in Personalunion mit dem assyrischen vereinigt. Ihrem Beispiele sind auch die Perserkönige gefolgt, bis Xerxes 478 Babylon zerstörte (s. u.) und dem babylonischen Königtum auch in diesem Sinne ein Ende machte. Ein direkt entgegengesetztes Verfahren sehen wir in älterer Zeit nur \emph{Tuklat-Ninib 1.} (um 1290\footnote{\emph{Zwei Hauptprobleme}, S. 61 ff.}) und später \emph{Sanherib} (689 v. Chr.) einschlagen, die Babylonien bei mehr oder minder radikalem Vorgehen gegen die Hauptstadt selbst zur assyrischen Provinz herabwürdigten, indem sie sei es das Kultbild des \emph{Marduk} selbst (so Sanherib) sei es doch dessen wesentliche Insignien (so \emph{Tuklat-Ninib}) nach Assur verbrachten. Bestand zu \emph{Adadnirari's 3.} Zeiten zugleich mit der tatsächlichen Oberherrschaft über Babylonien der "`Wunsch, diesem Verhältnis eine, im babylonischen Sinne, möglichst legitime Grundlage zu geben, so konnte die Einführung eines für das babylonische Staatsrecht maßgebenden Kultus als eine Förderung dieser Bestrebungen gelten.\footnote{\textsc{Winckler}, \emph{Geschichte Babyloniens und Assyriens}, S. 120.} Der Assyrerkönig wurde der lästigen Verpflichtung enthoben, alljährlich zu Neujahr nach Babylon zu pilgern, auch konnte dadurch die Begründung eines einheitlichen assyrisch-babylonischen Reiches, in dem der Nachdruck auf Assyrien lag, angebahnt und ausgedrückt werden. Dass \emph{Adadnirari 3.} ein derartiges Ziel bewusst verfolgte, beweist auch die gleichfalls aus Kalach stammende rein genealogische Inschrift des Königs. Nachdem er seine Genealogie bis zu seinem Urgroßvater \emph{Asur-nasir-abal 3.}\footnote{Als Dritter seines Namens ist \emph{Asurnâṣirabal}, der Vater \emph{Salmanassar's 2.}, erwiesen worden durch \textsc{Tiele}, ZA. 14. S. 392 f.} (885-60) geführt hat, greift \emph{Adadnirari 3.} hier, unter Übergehung selbst eines so bedeutenden Vorfahren wie \emph{Tiglatpileser 1.} auf den weit älteren König \emph{Tuklat-Ninib 1.} zurück. Man hat das verwunderlich gefunden, und Erklärungen versucht, die nicht befriedigen konnten.\footnote{Vgl. mit \textsc{Tiele}, \emph{Bab.-ass. Geschichte}, S. 210: \textsc{Winckler}, ZA. 2. 387 ff.} Offenbar will \emph{Adadnirari 3.} seine Abstammung von demjenigen Assyrerkönige besonders hervorheben der bisher allein, gleich ihm, Assyrien und Babylonien unter einem Zepter vollständig vereinigte. \emph{Tuklat-Ninib} hat nach der Eroberung Babylons 7 Jahre lang auch über Babylonien geherrscht.

Warum der Kult des \emph{Nebo}, nicht der des \emph{Marduk} selbst, von \emph{Adadnirari 3.} eingeführt wäre, würde sich durch die folgenden Erwägungen m. E. gleichfalls leicht erklären. Zwischen den Göttern \emph{Bêl-Marduk} von Babylon und \emph{Assur} resp. dem \emph{Bêl} von \emph{Assur} bestand eine, dem Gegensatz zwischen den durch sie repräsentierten und personifizierten Völkern und ihren Ländern entsprechende, Konkurrenz. Der Kult des \emph{Marduk} war nachweislich älter und, historisch wie religionsgeschichtlich, bedeutsamer, eine Tatsache, die durch die mehrfach zu beobachtenden Bemühungen der Assyrer, das Verhältnis umzukehren, nur bestätigt wird. \emph{Ašurbanabal}, der auf Anordnung seines Vaters \emph{Asarhaddon} das von \emph{Sanherib} entführte Kultbild des Marduk nach dem neugegründeten Babylon zurückführt und seinen Bruder \emph{Šamaššumukîn} zum König in Babylon einsetzt (668 v. Chr.), spricht von "`\emph{Marduk}, der während der Regierung eines früheren Königs"' (nämlich seines Grossvaters \emph{Sanherib}) vor dem Vater, der ihn erzeugt (dem Hauptgotte von Assur), sich in Assur niedergelassen hatte, und der nun wieder in Babylon einziehe.\footnote{\emph{Šamaššumukîn}, T. 1., S. 43.} Wollte man sich auf so wahrheitswidrige Behauptungen nicht einlassen und überhaupt feindseliges Vorgehen gegen Babylon und seinen Hauptgott vermeiden, andererseits aber auch dem Kult des \emph{Marduk} nicht noch eine weitere Stärkung und Verbreitung angedeihen lassen, so bot sich ein wirksamer Ausweg allerdings in der Einführung des \emph{Nebo}-Kults. Denn das Kultbild des \emph{Nebo} wurde zum Neujahrsfest regelmäßig in Prozession von Babylon nach Borsippa gebracht und \emph{Nebo} war dergestalt als Sohn des \emph{Marduk} gegenwärtig bei und indirekt beteiligt an der das babylonische Königtum bedingenden Zeremonie des "`Handerfassens."' Und wenn \emph{Nebo} in unserer Inschrift "`Sohn von \emph{Esaggil}"' genannt wird, so wird man das dem Bestreben zuschreiben dürfen, sein nahes Verhältnis zum babylonischen Hauptgottes im staatsrechtlichen Sinne zu betonen, ohne diesen selbst zu nennen. Ja, man ist noch weiter gegangen. Die Bezeichnung des \emph{Nebo} "`als Sohn des Gottes \emph{Nugimmud}"' d. h. des \emph{Ea} (\emph{Aë})\footnote{\textsc{Jastrow}, \emph{Religion}, S. 230.} in der der Einführung des \emph{Nebo}-Kults dienenden Inschrift (Z. 2) ist eine direkte Ketzerei, ein absichtlicher Verstoß gegen die offizielle babylonische Theologie, die \emph{Nebo} als Sohn des \emph{Marduk} (S. 261), und erst als Enkel des \emph{Ea}, des für die Menschen direkt nicht erreichbaren, zu fern und hochstehenden allweisen Heilsgottes betrachtet. Man griff damit auf ältere und ursprünglichere Vorstellungen zurück, in die Zeit, da Babylon noch nicht die erste Stadt Babyloniens und der Herrschersitz des geeinten Reiches war.\footnote{Danach ist \textsc{Jastrow} a. a. O. S. 125 zu berichtigen, der angibt: Da \emph{Marduk} als Sohn des \emph{Ea} bezeichnet wurde, "`so haben sich keinerlei Anzeichen einer ursprünglichen Verwandtschaft des \emph{Nabû} zu \emph{Ea} erhalten."'} Das geschah aus politischen, nicht aus antiquarischen Rücksichten. Man suchte \emph{Marduk} nach Möglichkeit auszuschalten, trug aber --- um völlige Klarheit, wie so oft in Theogonieen, unbekümmert --- Sorge, die Beziehung zum Tempel \emph{Esaggil}, in welchem nun einmal das babylonische Königtum erworben wurde, zu wahren. Wir werden alsbald sehen, dass auch in anderen Fällen beabsichtigten oder notgedrungenen Verzichts auf eine Berücksichtigung \emph{Bêl-Marduks} der Kultus des \emph{(Bêl-)Nebo} an dessen Stelle trat.

Wie die babylonischen Heiligtümer, so erhielt auch der neue assyrische Tempel den Namen \emph{Ezida}: dem \emph{Nabû}, der da wohnt in \emph{Ezida}, dem in Kalach belegenen, gelten Bild und Inschrift. Auf die durch diese Verpflanzung des \emph{Nebo}-Dienstes mit bedingte Gestaltung des staatsrechtlichen Verhältnisses Babyloniens zu Assyrien kommen wir noch zurück (S. 277). ---

\emph{Sammuramat} wird in keiner der direkt auf den Namen \emph{Adadnirar's 3.} lautenden offiziellen Inschriften genannt. Die \emph{Nebo}-Inschrift ist, wie schon die ständige Widmungsformel zeigt, mit Genehmigung des Königs gesetzt, also ein offiziöses Dokument. Man durfte daher einen Kausalnexus zwischen der Nennung der \emph{Sammuramat} gerade in dieser Inschrift und der Einführung des \emph{Nebo}-Dienstes mit einiger Wahrscheinlichkeit vermuten. Zur Voraussetzung eines solchen Zusammenhanges gezwungen werden wir aber erst durch Herodot, der sie als Beherrscherin von Babylon kennt.

Über die Identität der Personen lässt, wie schon von \textsc{G. Rawlinson} und \textsc{Ed. Meyer} betont, Herodots Zeitangabe keinen Zweifel. Sie ist zwar keineswegs genau, stellt aber unverkennbar eine leidliche Annäherung dar. Die Bauten, die Herodot, der \emph{Nitokris} zuschreibt, rühren in Wahrheit von \emph{Nebukadnezar} her, und es ist eine von mir selbst wie von Anderen erkannte,\footnote{\textsc{C. P. Tiele}, \emph{Bab.-ass. Gesch.} 2. 423; \textsc{C. F. Lehmann}, \emph{Berl. Phil. Wochenschr.} 1894, Sp. 272, 1898, 486, \emph{Wochenschr. f. klass. Phil.} 1900, 962; \textsc{Nikel}, \emph{Herodot und die Keilschriftforschung} (1896) S. 46; \textsc{Ed. Meyer}, \emph{Forschungen zur alten Geschichte} 2. (1899) 478 f. Anm. 1.} unten noch weiter zu erklärende Tatsache, dass die vermeintliche babylonische \emph{Nitokris} nur einem Missverständnis, einer Verstümmelung des Namens, persisch \emph{Nabukadracara}, ihre Entstehung verdankt. Von \emph{Nebukadnezar's 2.} Regierungs beginn (605) an --- also besonders günstig --- 5 herodoteische\footnote{Hekataios rechnete die Generation m. E. zu 35 Jahren, s. \emph{Hermes} 35., S. 649.} Generationen aufwärts rechnend, kommen wir auf 772; die Mitte seiner Regierung zum Ausgangspunkt nehmend, auf ca. 755.

Diese \emph{Sammuramat}, der wir in Assyrien begegnen und die gleichzeitig Babylonien beherrscht, ist unter den von uns ermittelten Verhältnissen nur denkbar als Gemahlin eines Assyrerkönigs und zwar nur als Gemahlin \emph{Adadnirari's 3.}

Für die von verschiedenen Seiten vertretene oder in Betracht gezogene Annahme, dass sie dessen Mutter, die Gemahlin \emph{Šamši-Adads} gewesen wäre, bleibt keinerlei Raum, seitdem wir wissen (S. 263), dass \emph{Šamši-Adad} bis zu seinem letzten Jahre mit Babylonien in Feindschaft gestanden hat und dass ihm \emph{Adadnirari 3.} als sein offenbar großjähriger Sohn gefolgt ist. Die Art wie \emph{Bêl-tarṣi-iluma} in der \emph{Nebo}-Inschrift den \emph{Adadnirari} und die \emph{Sammuramat} als "`seinen Herrn"' und "`seine Herrin"' nebeneinander nennt, bestätigt diesen Schluss.\footnote{Dass \emph{Sammuramat} die Mutter \emph{Adadnirari's 3.} gewesen sei, nahm namentlich \textsc{Hommel}, \emph{Geschichte} a. a. O. (vgl. \textsc{Tiele} u. \textsc{Maspero} a. a. O.) an, aber aus unzulässiger Rücksicht auf den Zug der Sage, dass \emph{Semiramis} beim Tode des \emph{Ninos} die Herrschaft für ihren unmündigen Sohn übernommen hätte. Hiervon ist \textsc{Hommel}, jetzt (\emph{Ein neuer babyl. König}, S. 20 Anm. 22) zurückgekommen. --- Die sonst m. W. nicht zum zweiten Mal belegte Bezeichnung "`Frau des Palastes"' findet sich merkwürdiger Weise in der bei \textsc{Goethe}, \emph{Westöstlicher Divan} (S. 362 der Hempelschen Ausgabe) wiedergegebenen Übersetzung des \emph{Schreibens der Gemahlin des Kaisers von Persien an Ihre Majestät die Kaiserin Mutter aller Reußen} zur Bezeichnung der Letzteren. Da es aber, wie mir auch \textsc{Noeldeke} und \textsc{F. C. Andreas} bestätigen, unerfindlich ist, wie dieser Ausdruck im Persischen gelautet haben sollte, so wird ein Irrtum des Übersetzers vorliegen.} Als Witwe des 783 verstorbenen \emph{Adadnirari} könnte \emph{Sammuramat} 772 allenfalls sogar noch gelebt haben.

Wie aber kommt es, dass Herodot in Babylon und als für Babylon speziell bedeutsam eine Herrscherin nennt, die in Wahrheit doch eine Assyrerkönigin war, eine von den nicht wenigen babylonischen Prinzessinnen, denen das gleiche Loos zu Teil wurde?\footnote{Außer den S. 264 erwähnten Fällen ist namentlich hinzuweisen auf die Babylonierin, die \emph{Assarhaddon} (681-68) neben seiner assyrischen Gemahlin zur legitimen Frau erkor und die ihm den \emph{Šamaššumukîn} gebar, \emph{Zwei Hauptprobleme}, S. 104 u. 209.} Man halte uns nicht entgegen, dass Herodot sie ja unter Bezugnahme auf die nicht zur Ausführung gelangten\footnote{Der jetzt wieder von \textsc{Ed. Meyer} vertretenen Ansicht (\emph{Forschungen} 2. 198 f. Anm. 1), dass die Ἀσσύριοι λόγοι ein gesondertes Werk hätten bilden sollen, vermag ich mich, auch aus obigem Grunde, nicht anzuschließen. Der Änderung im Plane des herodoteischen Werkes ist der beabsichtigte Exkurs über die babylonische Geschichte zum Opfer gefallen. Dieser Exkurs konnte sehr wohl in späteren Teilen des Werkes, z. B. gelegentlich des "`von Zopyros"' bewältigten babylonischen Aufstandes seinen Platz finden.} Ἀσσύριοι λόγοι nenne; sie könne also, auch nach Herodots Information, eine assyrische Königin gewesen sein, die gleichzeitig auch über Babylonien geherrscht habe, und es sei somit nicht einmal sicher, dass er sie als eine Babylonierin habe bezeichnen wollen. Denn, wie ich bereits einmal ausgesprochen habe\footnote{\emph{Sitzungsber. archäol. Ges.} Nov. 1895 = \emph{Wochenschr. f. klass. Phil.} 1896 No. 3 Sp. 84 f.} und in anderem Zusammenhange nochmals ausführlicher zu begründen gedenke,\footnote{In meinen \emph{Forschungen zu Herodot und Hekataios}, in denen Vieles hier und andernorts nur Berührte zu seinem vollen Rechte kommen soll und deren ursprüngliches Manuskript demnächst bereits das horazische Alter erreicht haben wird.} über den beabsichtigten Inhalt dieser Ἀσσύριοι λόγοι macht man sich allgemein eine ganz irrige Vorstellung. Für Herodot ist Babylon die Hauptstadt von "`Assyrien,"' aus dem einfachen Grunde weil in der Satrapieeneinteilung des Darius Assyrien und Babylonien eine Satrapie bildeten.\footnote{S. die Belege und meine Bemerkungen, \emph{Wochenschr. f. klass. Phil.} 1900, Sp. 962 f. Anm. 6.} Dieses Verhältnis hat zwar wahrscheinlich nur bis auf Xerxes\footnote{Die Veränderung wurde veranlasst durch den von mir nachgewiesenen zweiten Aufstand der Babylonier gegen Xerxes unter Führung des Usurpators \emph{Tar(\U{H}az)-zi-a}, s. meinen Aufsatz "`\emph{Xerxes und die Babylonier},"' \emph{Wochenschr. f. klass. Phil.} 1900, Sp. 959 ff. und dazu \textsc{Ed. Meyer}, GA. 3., \emph{Nachträge und Berichtigungen} zu § 86. --- \textsc{Weissbach}, ZDMG. 55. (1901) S. 209, holt \textsc{Pinches}' unhaltbare Identifikation dieses \emph{Tar(\U{H}az)-zi-a} mit \emph{Bar-zi-ia} (-Smerdis) wieder hervor. Das Täfelchen ist datiert vom 11./8. des 1. Jahres, Bar-zi-ia aber ist im siebenten Monat seines ersten Jahres ermordet worden. \textsc{Weissbach} meint: "`da aber diese Ermordung im fernen Medien erfolgt war, so ist es fraglich, ob die Kunde davon sogleich in alle Orte Babyloniens drang."' Dieser an sich sehr fragwürdige Notbehelf verbot sich von vornherein: die Urkunden des Usurpators "`\emph{Nebukadnezar 3.},"' der dem \emph{Barziia} in Babylon folgte, beginnen, wie allbekannt und zudem ausdrücklich von mir hervorgehoben, bereits mit dem 17./7. des Antrittsjahres. Es bleibt bei dem zweiten Aufstand des \emph{Tar(\U{H}az)-zi-ia} 479/8, der wahrscheinlich Xerxes' Rückkehr aus Sardes veranlasste und die Zerstörung der Haupttempels \emph{Esaggil} sowie der äußeren Mauern von 480 Stadien Umfang und das Ende des nominell in Personalunion mit dem persischen weiterbestehenden babylonischen Königtums zur Folge hatte.} bestanden. Aber Herodot bewegt sich nachweislich gerade in seinem Bericht über Babylon und Babylonien vielfach in den Vorstellungen seiner Quellen, der Logographen\footnote{Besonders Dionysios von Milet, der, wie überhaupt, so auch speziell für die babylonischen Nachrichten, als eine der das Selbstgesehene ergänzenden und beeinflussenden schriftlicher Quellen Herodots zu betrachten ist. Vgl. \emph{Wochenschr. f. klass. Phil.} 1900, S. 964 Anm. 1 u. 6. Da Dionysios natürlich seinen von Herodot auch direkt verwerteten Landsmann Hekataios benutzt hat, so ergibt sich schon hier ein ziemlich verwickeltes, aber für die babylonischen Nachrichten doch großenteils entwirrbares Quellenverhältnis. An der Ansicht, dass Strabo 16., 1, 14. 20 Hekataios' Schilderung der babylonischen Sitten und Gebräuche nur sprachlich modifiziert wiedergibt und dass uns so die hekatäische Vorlage der entsprechenden Abschnitte bei Herodot (1., 193 ff.) erhalten ist, halte ich (\emph{Festschrift für Heinrich Kiepert} S. 305 ff.) gegen \textsc{Eduard Meyer} (\emph{Forschungen zur alten Geschichte} 2., 233 Anm. 1) fest. Als eine Erweiterung und Überarbeitung der Daten Herodots nach den Anschauungen der späteren Zeit, können dieser und verwandte Abschnitte bei Strabo gerade nicht entstanden sein und verstanden werden. Er stimmt vielmehr "`zu der Eigenart"' der Hekataios, "`wie sie"' (so ließ \emph{Kiepert-Festschrift} S. 307 Z. 10) "`namentlich durch \textsc{Diels} festgestellt worden ist,"' und Dinge, die bei Herodot unverständlich sind erklären sich durch seine bei Strabo erhaltene Vorlage. Näheres demnächst. Vgl. vorderhand auch \emph{Hermes}, 35. S. 649 u. Anm. 4.} aus der Zeit des Darius und der ersten Jahre seines Nachfolgers. Herodots Nachrichten bilden hier ein sehr eigentümliches Gemisch von Übernommenem und Selbstgeschautem, das jedoch der Hauptsache nach ganz wohl entwirrt und in seiner Entstehung verfolgt werden kann. So behält er auch --- und nicht er allein\footnote{Auch Xenophon, der in der Anabasis Babylonien sehr wohl von Assyrien resp. "`Syrien zwischen den Flüssen"' zu unterscheiden weiß, bezeichnet in der Cyropädie den von Cyrus bekriegten Beherrscher Babylons und Babyloniens als Ἀσσύριος. Dies erklärt sich aus Benutzung einer älteren Quelle, der er u. A. auch die Kenntnis der historischen Rolle des Gobryas-\emph{Ugbaru} (sowie ferner beispielsweise der durchaus sachgemäß geschilderten Kämpfe zwischen Armeniern und Chaldern [vgl. \emph{Verh. Berl. anthrop. Ges.} 1895, S. 585 ff. u. Anm. 1]), verdankt. Der für die Logographenzeit berechtigte Sprachgebrauch wirkt in der im Altertum bei den Späteren vielfach herrschenden Verwirrung der Begriffe nach.} --- die für jene frühere Zeit berechtigte Terminologie bei. Und deshalb bezeichnet er die Nachrichten, die er in und über Babylon und Babylonien eingezogen hat, als Ἀσσύριοι λόγοι. Von Assyriens Geschichte im eigentlichen Sinne weiß er so gut wie nichts weiter, als dass Niniveh zerstört worden ist.\footnote{Her. 1. 106, 178, 185. Außerdem kennt er Züge der Sardanapal-Legende 2. 150 (ᾔδεα λόγῳ).} Nur in Ägypten hat er außerdem vom beabsichtigten Angriffe des Sanherib, der, vom ägyptischen Standpunkt ganz richtig, als βασιλεὺς Ἀραβίων τε καὶ Ἀσσυρίων bezeichnet wird (2. 141), Kunde erhalten. Dass die Ἀσσύριοι λόγοι zum ersten Mal in Verbindung mit Ninivehs Fall erwähnt werden, hat die richtige Einsicht erschwert. Es geschieht aber --- wie mau erkennen muss, sobald man sich die Bedeutung der Begriffe Assyrien und Babylonien bei Herodot in diesem gesamten Zusammenhange klar macht --- nur deshalb, weil die Babylonier am Falle Assyriens und Ninivehs wesentlich beteiligt waren,\footnote{Die wiederholten neueren Versuche, den Anteil der Babylonier zu leugnen oder als möglichst gering hinzustellen, haben an dieser Herodot-Stelle daher gewiss keine Stütze. Vgl. auch ZA. 14. 335, Anm. 3.} worauf Herodot (1. 107) auch hindeutet mit den Worten, dass die Meder τοὺς Ἀσσυρίους ὑποχειρίους ἐποιήσαντο πλὴν τῆς Βαβυλωνίης μοίρης: Assyrien mit Ausnahme des Babylonien genannten Teiles wurde unterworfen. Also die Ἀσσύριοι λόγοι würden spezifisch babylonische Nachrichten enthalten haben. Und es bleibt zu ermitteln, welche besonderen Umstände dazu führten, dass Herodot in Babylon von der Semiramis erfuhr.

Auch der Inhalt seiner Nachrichten gibt darüber zunächst keinen Aufschluss. Sie sind freilich schon deshalb wertvoll, weil sie von jedwedem Anklang an die Heldin des Romans frei sind. Aber die Regulierung der Wasserverhältnisse: Entwässerung, Kanalisation, Aufführung von Dämmen und die Instandhaltung dieser Anlagen, gehören zu den Hauptaufgaben jedes babylonischen Königs, der es mit seinem Herrscheramte irgendwie ernst nimmt. Sie bilden die unerlässliche Voraussetzung für die Bewohnbarkeit und Ertragsfähigkeit des Landes, und gleich der Begründer des einheitlichen semitischen Reiches mit der Hauptstadt Babylon, \emph{\U{H}ammurabi} (um 2230\footnote{\emph{Zwei Hauptprobleme} 105 ff., 118, verglichen mit \textsc{Marquardt}, \emph{Chronologische Untersuchungen} S. 649[16] ff.}), sucht einen Ruhmestitel in seiner Fürsorge auf diesem Gebiete.\footnote{Anlage des "`\emph{\U{H}ammurabi-Kanals},"' "`des Segens der Menschen, der da reichliches Wasser bringt dem Lande..."' KB. 3. 1, S. 122.} Ein gleiches haben wir von vielen babylonischen Königen aller Zeiten, für deren eigentliche Regierungstätigkeit unser Material versagt, vorauszusetzen. Über \emph{Nebukadnezars 2.} Bemühungen um Wasserbau und Kanalisation liegen urkundliche Berichte vor.\footnote{S. große Steinplatteninschrift Col. 6., 39 ff., ferner des Königs Kanal-Inschrift KB. 3. 2, S. 60. Vgl. Berosos und Abydenos.} Von diesen Anlagen hat auch Herodot Kunde (1. 185 sq.). Aber diese Wasserbauten (der "`Nitokris"') werden von ihm mit mannigfaltigen und staunenerregenden Einzelheiten geschildert. Was er dagegen über die \emph{Semiramis} berichtet, ist als Ganzes und im Einzelnen so wenig unterscheidend, so farblos, dass man die verwunderte Frage wiederholen muss: wie kommt Herodot zur Kenntnis der in nüchterner Realität dastehenden, jeglicher romanhaften Ausschmückung entbehrenden Herrscherin \emph{Semiramis}?

Die Antwort wird durch die engere Bestimmung der Stelle, an der Herodot seine Erkundigungen eingezogen hat, erschlossen. Aus der kritischen Betrachtung seiner Schilderung von Babylon ergibt sich nämlich, dass er nicht, wie er selbst glaubte, den Tempel des \emph{Bêl-Marduk}, \emph{Esaggil}, in Babylon besucht hat, sondern den des (Bêl-)\emph{Nebo} in \emph{Borsippa}. Zu diesem Schlusse zwingen, wie ich bereits mehrfach angedeutet habe,\footnote{\emph{Berl. Phil. Wochenschr.} 1894, Sp. 271 f., 1898, Sp. 485, \emph{Wochenschr. f. klass. Phil.} 1900, S. 964 f. Anm. 6. Unabhängig von mir kam zu demselben Ergebnis \textsc{Nikel}, \emph{Herodot und die Keilschriftenforschung} 1896, S. 27, 29 ff. (Vgl. S. 270 Anm. 5.)} namentlich zwei Umstände. Einmal lag \emph{Esaggil} auf dem linken Euphratufer, demselben auf welchem die große Königsburg lag, deren Ruinen unter dem Schutthügel des "`\emph{\d{K}asr}"' begraben sind. Dieses längst aus den Inschriften \emph{Nebukadnezars} mit voller Bestimmtheit erkannte Resultat\footnote{S. namentlich die große Inschrift Nebukadnezars, Col. 8. 31 ff., (KB. 3. 2 S. 26/27).} ist zum Überfluss noch durch die deutschen Ausgrabungen in Babylon\footnote{Auffindung der Prozessionsstraße des Gottes \emph{Marduk}, von der \emph{Nebukadnezar} (s. vorige Anmerkung) spricht, sowie der Ruinen des Tempels \emph{Esaggil} selbst (im Hügel \emph{Tel Amran ibn Ali}), s. den \emph{Zweiten} und \emph{Dritten Jahresbericht der Deutschen Orientgesellschaft} und deren \emph{Mitteilungen} No. 5-7.} bestätigt\footnote{Irgendein Novum für die Beurteilung des Herodot liefern diese Ausgrabungsergebnisse mit Nichten.} worden. Der von Herodot besuchte Tempel dagegen lag auf der dem Königspalast entgegengesetzten Seite, also rechts des Euphrat. Er hatte einen Stufenturm. Auf der rechten Euphratseite aber von Babylon aus bequem erreichbar und, in das äußere Mauerviereck von 480 Stadien Umfang seinerzeit mit einbezogen, also auch deshalb (vgl. S. 267 f.) zu Babylon im weitesten Sinne gehörig,\footnote{Das große Babel \emph{Nebukadnezars} ist eine Doppelstadt gewesen, insofern \emph{Borsippa} in die äußere Mauer von 480 Stadien Umfang einbezogen gewesen sein muss. Dies ergibt schon die Rekonstruktion des Verlaufs der Mauern nach \textsc{Oppert}, besonders wenn man das richtige Maß für das babylonisch-persische Stadium (7 1/2 auf die römische Meile) zu 198,47 m (s. \emph{Actes du huitième Congrès des Orientalistes. 2\textsuperscript{me} partie.} p. 195 ff., 229 ff., 242 f. und Tabelle) einführt. --- Auch wenn sich \textsc{Opperts} Plan, was den Verlauf der Mauern anlangt, als modifikationsbedürftig erweisen sollte, so wird sich schwerlich eine Lage des äußeren Mauerquadrats von 480 Stadien Umfang ergeben und finden lassen, in welche Borsippa nicht mit einbegriffen wäre. Nach Mitteilungen, die von der Ausgrabungsstätte der Deutschen Orient-Gesellschaft in Babylon in die Öffentlichkeit dringen (s. \textsc{Rohrbach}, "`\emph{Babylon}."' \emph{Preussische Jahrbücher} Bd. 104 (1901), S. 276), scheint die Ansicht Platz zu greifen, dass die Nachrichten der Alten über Umfang und Größe der Stadt ins Reich der Fabel gehörten, dass Babylon niemals entfernt die Größe gehabt habe, die ihm nach den bisherigen Vorstellungen zukomme. Es ist wohl anzunehmen, dass die topographischen Untersuchungen im weiteren Umkreis des Stadtkernes noch nicht zum Abschluss gekommen sind. Und jedenfalls erscheint es mir dringend geboten, einmal wieder darauf hinzuweisen, dass bekanntermaßen die antike Tradition über die Größe der Stadt durchaus nicht auf Herodot allein beruht und daher auch nicht mit geringschätzigen Seitenblicken und Zweifeln an dessen Glaubwürdigkeit abgetan werden kann. Zunächst lässt sich mit Sicherheit nachweisen, dass derartige genaue Maßangaben, noch dazu, wie die geraden sexagemisalen Zahlen beweisen, in den ursprünglichen babylonisch-persischen Maassen, niemals auf Herodot selbst zurückgehen, der Zahlen- und Maaßverhältnissen nur ein sehr geringes Verständnis und äußerst primitive Vorstellungen entgegenbrachte, sondern von ihm aus logographischer älterer Quelle (Hekataios, für den die Beachtung der metrischen Verhältnisse genügend bezeugt ist), zurückgehen, ebenso der Vergleich der gemeinen (bab.-pers.-pheidonischen) und der königlichen Elle. Näheres in meiner Untersuchung: "`\emph{Die metrischen Angaben bei Herodot als Stützpunkte für die Kritik}"' (vgl. S. 270 Anm. 5). Damit stimmt es, dass dieser äußere Mauerzug durch Xerxes zerstört worden war (s. \emph{Wochenschr. f. klass. Phil.} 1900, Sp. 965 Anm. 4; oben S. 271 Anm. 1), sodass Herodots Angaben hier nur auf eine ältere Quelle zurückgehen können. Dass er sie gleichwohl als unversehrt schildert, stimmt zu dem Bilde, das wir an dieser und an mancher anderen Stelle, von Herodots Arbeitsweise erhalten (gegen \textsc{Peiser}, \emph{Studien zur orientalischen Altertumskunde} 3. 25, \emph{Mitteil. der vorderas. Gesellsch.} 1900, No. 2). Über \emph{Plinius}' und \emph{Solinus}' auf dasselbe hinauskommende, m. E. auf Hekataios beruhende Daten s. \emph{Congressakten} a. a. O. S. 232 Anm. 1 u. S. 233 Anm. 1. --- Die auch Herodot bekannte innere Mauer musste danach allein stehen bleiben. Sie hatte nach dem Zeugnis des Ktesias und der Alexanderhistoriker einen Umfang von 360 Stadien, woraus mit leicht erklärlicher Modifikation bei einigen 365 wird. Wenn Curtius (5., 1) ausdrücklich angibt, dass die Mauer 360 (365) (so ließ statt 368) Stadien Umfang gehabt habe, dass aber von dem von ihr umschlossenen Gebiet nur ein Areal von ca. 90 Stadien Umfang wirklich bewohnt gewesen ist, so ist es (gegen \textsc{Rohrbach}) natürlich ganz unmethodisch, den letzteren Teil der Angabe für richtig zu halten, den ersteren einfach unbeachtet zu lassen. Und wer die Daten bei Curtius nach Klitarch wegen ihrer Verwandtschaft mit den ktesianischen Daten über die Mauern der von der Semiramis "`gegründeten"' Stadt Babylon (Diod 2. 7) beargwöhnen möchte, der ist darauf hinzuweisen, dass, wie ich, \emph{Wochenschr. f. klass. Phil.} 1895, Sp. 184 dargetan habe, bei \textsc{Strabo} und sonst eine Umrechnung der babylonisch-persischen Maße in ptolemäisch-ägyptische vorliegt, die nur auf Ptolemaios 1. und seine Alexandergeschichte zurückgehen kann. Auch Berosos kommt bekanntlich in Betracht. Sollten die weiteren Forschungen an Ort und Stelle ergeben, dass wirklich nirgends Spuren der quadratischen Mauern von 480 und 360 Stadien Umfang sich erhalten haben, so wird immer noch zu fragen sein, ob diese ungeheuren Mauern nicht vielleicht von einer Struktur gewesen sind, durch die sich ein spurloses Verschwinden derjenigen Reste, welche feindliche Zerstörung und die Verwendung der besseren Bestandteile des Materials für anderweitige Bauten übriggelassen, erklären würde. Es ist dabei zu bedenken, dass diese Mauern die Stadt mit weiten Teilen ihres Gebietes umschlossen, während in der eigentlichen bewohnten Stadt wieder gewisse Teile besonders ummauert waren.} lag Borsippa mit \emph{Ezida} und dem zugehörigen Stufenturm (s. o. S. 261). Andere Stufentürme können nicht in Betracht kommen, da jede babylonische Stadt (im engeren Sinne) nur einen solchen zu ihrem Haupttempel gehörigen Bau, Festungswerk und Observatorium zugleich, besaß.

Zum anderen war Babylons Haupttempel \emph{Esaggil} von \emph{Xerxes} zerstört worden.\footnote{Die Belege s. \emph{Wochenschr. f. klass. Phil.} 1900, S. 964, Anm. 4-6. Nach Arrian 7., 17, 2 wurde der Tempel von Xerxes zerstört, als er aus Griechenland zurückkehrte. \textsc{Ed. Meyer} (\emph{Forschungen} 2., 478) bezweifelte dies, da der eine babylonische Aufstand gegen Xerxes, mit dem er rechnete, vor dem Zuge gegen Griechenland erfolgt sei. Da \textsc{Meyer} (vgl. S. 271 Anm. 1) den von mir nachgewiesenen zweiten Aufstand anerkennt, der Xerxes vorzeitig vom Griechenkriege zurückrief, so ist wohl auch sein Zweifel an der Richtigkeit der vielen die Zerstörung des Belstempels durch Xerxes berichtenden klaren Zeugnisse als aufgegeben zu betrachten.} Nach dem, was oben (S. 268) über die engen Beziehungen zwischen \emph{Marduk} und \emph{Nabû} dargelegt wurde, leuchtet ein, dass in die Funktionen des zerstörten Hauptgottes und Haupttempels \emph{Nebo} und sein Haupttempel \emph{Ezida} in Borsippa, wenigstens vorläufig bis anderer Ersatz geschaffen war,\footnote{Auch wenn ein solcher notdürftiger Ersatz geschaffen wurde, wofür Einiges zu sprechen scheint, wird doch für mancherlei Äußerlichkeiten der Kultus, besonders für die großen Feste, die Stellvertreterschaft des unversehrten alten Nebotempels gewahrt geblieben sein.} einrückten.\footnote{Im Jahre 268 begann \emph{Antiochos 1.} die Erneuerung der Tempel \emph{Esaggil} in Babylon und \emph{Ezida} in Borsippa (KB. 3. 2, S. 136), die er schon im Jahre 274/3 (ZA. 6., S. 236 Z. 40) "`Ziegel für den Bau von \emph{Esaggil} [und \emph{Ezida}"' (so durch den Raum auf dem Original gefordert)] "`wurden oberhalb und unterhalb Babylons gestrichen"' in Angriff genommen hatte und deren Ausführung durch den ersten syrischen Krieg verhindert worden war. Diese Erneuerung bedeutete für \emph{Ezida} eine Restauration, für den babylonischen Haupttempel dagegen ein vollständiger Wiederaufbau, eine Wiederaufnahme von Alexanders unausgeführtem Projekt. Bis zur Durchführung dieses Neubaues blieben \emph{Nebo} und sein Tempel im Vordergrund. So erklärt sich in der uns erhaltenen Inschrift des Königs aus dem Tempel in Borsippa, die starke Betonung der Sohnes-Qualität des \emph{Nebo}: \emph{Nabû}, Sohn \emph{Esaggils} (vgl. o. S. 268), erstgeborner Sohn des \emph{Marduk}, Kind der \emph{Erûa} "`der Königin"' (d. i. \emph{Marduks} Gemahlin), Col. 2. Z. 4 ff.; "`\emph{Nabû}, erstberechtigter (\emph{ašaridu}) Sohn"' Z. 21 f.}

Und nun erklärt sich, wie selbstverständlich, Herodots Kunde von der \emph{Semiramis}: die Priesterschaft des \emph{Nebo} von \emph{Borsippa} musste der Herrscherin, zu deren Zeit und unter deren Mitwirkung ihrem Kult eine so entscheidende Förderung und Ausbreitung erwuchs, naturgemäß ein dauernd dankbares Andenken bewahren. So erweist sich einmal der Schluss als zutreffend, der schon aus der Erwähnung der \emph{Sammuramat} in der Einführungsinschrift gezogen war: der Babylonierin \emph{Semiramis}, der Gemahlin des Assyrien und Babylonien beherrschenden \emph{Adadnirari 3.}, kommt ein wesentlicher Anteil an dieser religiös-politischen Maßregel zu. Zweitens aber --- und das ist das wichtigste --- bestätigt sich die auf ganz anderem Wege, ohne jede Berücksichtigung der herodoteischen \emph{Semiramis}-Nachricht und ihrer Herkunft gewonnene Erkenntnis, dass Herodot den \emph{Nebo}-Tempel zu \emph{Borsippa} besucht und dort seine Erkundigungen eingezogen hat.

Unter diesem Gesichtspunkt erhält denn auch seine Nachricht über die Wasserbauten der \emph{Semiramis} Farbe und Leben. Noch heute liegt, was von Borsippa übrig ist, das Fundament des Nebotempels, "`\emph{Birs-Nimrud},"' in oder nahe dem Überschwemmungsgebiet eines alten Euphratarms, des \emph{Hindîye}. Für die Förderin des \emph{Nebo}-Dienstes war eine etwa damals (wieder) notwendige Regulierung der Strom- und Bewässerungsverhältnisse um dessen Haupttempel eine lockende Aufgabe.

Auch die nun einmal bei Herodot zweifellos vorliegende Umwandlung des \emph{Nebukadnezar} in eine \emph{Nitokris} gewinnt so erheblich an Verständlichkeit. Von vornherein auf einen Vergleich Ägyptens und Babyloniens, den er zudem schon bei seinem Vorgänger Hekataios\footnote{Gerade da, wo sich Herodot auf die Bewohner der von ihm besuchten Länder nachdrücklichst beruft und ev. gegen sie polemisiert, ist anzunehmen und mehrfach erweislich, dass die erste Ermittlung nicht auf ihn, sondern auf Hekataios zurückgeht (vgl. \textsc{Diels}, \emph{Hermes} 22., 421 f., 436). So wird auch Her. 1. 182 der Vergleich babylonischer von den Chaldäern mitgeteilter mit ägyptischerseits bezeugten Vorstellungen (ὡς λέγουσι οἱ Αἰγύπτιοι) zu beurteilen sein.} fand, ausgehend und der ägyptischen \emph{Nitokris} eingedenk,\footnote{Herodot ist sicher in Babylonien später gewesen als in Ägypten. Dass er 2. 100 bei der ägyptischen Nitokris, die im ersten Buche behandelte "`Babylonierin"' als bekannt voraussetzt, bringt der Gang seiner Darstellung mit sich.} erhält er durch die \emph{Nebo}-Priester Kunde von der \emph{Semiramis} und hört sodann die Urheber der bedeutendsten Werke und Anlagen Babyloniens\footnote{Bei der Anlage des Beckens von Sippar, wie sie Her. 1. 185 schildert, erinnert der Ausdruck βάθος μὲν ἐς τὸ ὕδωρ ἀεὶ ὀρύσσουσα gleichfalls (vgl. o. S. 258 f. Anm. 5) an eine in den babylonischen Bauinschriften häufige Wendung: "`ich erreichte das Grundwasser"' (\emph{šupul mê akšud}) so u. A. auch bei \emph{Nebukadnezar}, Grosse Steinplatteninschr., Col. 7. 60.} mit einem Namen bezeichnen, der nicht nur mit \emph{N} anklingt, sondern --- wenn auch in anderer Reihenfolge --- die Konsonanten des Namens \emph{Nitokris} sogut wie sämtlich enthält, persisch \emph{Nabukadracara},\footnote{Babylonisch \emph{Nabûkudurruṣur}: Herodots Dolmetscher wird aber ein Perser gewesen sein. Vgl. auch \textsc{Ed. Meyer}, \emph{Forschungen} 1. 194.} im Sprechen von \emph{\textbf{N}abu\textbf{k}a\textbf{tr}a\textbf{c}ara} nicht zu unterscheiden. Flugs erkennt er in dem Namen einen alten Bekannten, den Frauennamen \emph{Nitokris}, und gesellt so der \emph{Semiramis} in der babylonischen \emph{Nitokris}\footnote{Für die phantastische Erzählung vom Grabe der \emph{Nitokris} Her. 1. 187 haben verschiedene Elemente die Grundlage gebildet. Die wichtigste der missverständlich verwerteten Tatsachen ist das Eindringen des Xerxes in die Mysterien des toten Bel beim Neujahrsfest des Jahres 484, s. \textsc{C. F. Lehmann}, \emph{Berl. Phil. Wochenschr.} 1898, 486, \emph{Wochenschr. f. klass. Phil.} 1900, 962 Anm. 1; \textsc{Ed. Meyer}, \emph{Forschungen} 2. (1899) 478 Anm. 1. Näheres demnächst.} einen weiblichen Nachfolger späterer Zeit zu. Wie leicht Namen einer Fremdsprache, auch wenn man deren nicht nicht ganz unkundig ist, missverstanden werden, weiß jeder Forschungsreisende aus eigner Erfahrung. Die herodoteische Umgestaltung zeigt die Merkmale einer Volksetymologie, nur dass wir diesmal das Individuum kennen, in dessen Phantasie sie sich vollzogen hat und dass sie sich an einen ihm bekannten fremden Eigennamen, nicht an ein Wort seiner Muttersprache anlehnt.\footnote{Dass der (medischen) Gemahlin \emph{Nebukadnezars} und dessen Tochter, der Gemahlin \emph{Nergalšaruṣur(-Neriglissar)'s}, als Mittelgliedern bei dieser Umformung eine Rolle zukomme, scheint mir wenig wahrscheinlich, wenn auch nicht undenkbar.}

Ein scheinbarer Widerspruch gegen Herodot scheint nun bei Berosos\footnote{An sich hätte in zweiter Linie die Möglichkeit in Betracht kommen können, dass τῆς Ἀσσυρίας nicht mit zu dem Citat aus Berosos gehöre, sondern Josephus' Bezeichnung wäre, in welchem Falle sie lediglich die landläufigen Vorstellungen seiner Zeit wiedergäbe. Da aber, wie im Text zu zeigen, das Attribut in Berosos' Munde einen sehr guten Sinn gibt, so braucht mit dieser sekundären Möglichkeit nicht weiter gerechnet zu werden.} vorzuliegen, der nach Josephus (S. 259) die Semiramis als Ἀσσυρία bezeichnete --- aber auch nur ein scheinbarer. Erklärlich wäre es schon, wenn Berosos, (da er gegen die späteren griechischen Autoren polemisierte, die die Assyrerin, die Gemahlin des Ninos, als Schöpferin der Wunderbauten Babylons hinstellten), die \emph{Semiramis}, unbekümmert um ihre babylonische Nationalität, als Gemahlin des assyrischen Fremdherrschers hätte charakterisieren wollen. Aber auch die nackte Tatsache, dass die \emph{Nebo}-Priester zur Zeit Herodots die \emph{Semiramis} als babylonische Königin anerkannten, Berosos (und seine Priesterklasse) zur Zeit Alexanders und der ersten Seleukiden dagegen nicht, ist begreiflich nicht nur, sondern fördert unser Verständnis der Sachlage. An solche Verschiedenheit der Auffassungen gerade gegenüber den assyrischen Machthabern, die faktisch Babylonien mit beherrscht haben, sind wir gewöhnt. Die babylonische Königsliste verzeichnet die acht Jahre der Herrschaft \emph{Sanheribs} nach Babylons Zerstörung, während der auf babylonische Daten zurückgehende ptolemäische Canon an entsprechender Stelle\footnote{\emph{Zwei Hauptprobleme}, S. 31 f.} bekanntlich eine achtjährige königslose Zeit anführt. Dass sich die Königsliste in unversehrtem Zustande \emph{Tuklat-Ninib 1.} gegenüber entsprechend verhielt, ist wahrscheinlich, aber da sie kein nach einheitlicher Auffassung redigiertes Dokument zu sein braucht, nicht sicher auszumachen.\footnote{Ebenda und S. 102, 141 f.} \emph{Adadnirari 3.} suchte mit milderen Mitteln dasselbe zu erreichen wie die beiden Genannten (S. 267 ff.), eine Vereinigung Babyloniens mit Assyrien, ein Aufhören des spezifisch babylonischen Königtums. Die von ihm in seinem Antrittsjahr den babylonischen Göttern dargebrachten Opfer (S. 263, 266) und nachmals die Einführung des \emph{Nebo}-Kults sollten nur dazu dienen, die Veränderung vorzubereiten und sie zu verschleiern. Daher vermeidet auch trotz aller scheinbaren Hinneigung zum babylonischen Wesen \emph{Adadnirari 3.} sich als babylonischen König, als "`König von Sumer und Akkad"' zu bezeichnen --- eine zuerst von \textsc{Tiele} \footnote{\emph{Geschichte} 1. 213.} hervorgehobene, aber nicht richtig beurteilte Tatsache. Staatsrechtlich, sowohl vom streng babylonischen als vom assyrischen Standpunkt, wie er spätestens von der Einführung des \emph{Nebo}-Kults ab erkennbar ist, gab es kein babylonisches Königtum; diesen strengen Standpunkt spiegelte Berosos' Bezeichnung wieder. Schon diejenigen aber, die das rein faktische Verhältnis ins Auge fassten und mehr noch solche, die daraus Vorteil zogen, konnten \emph{Adadnirari} als babylonischen König bezeichnen; seine Bemühungen der babylonischen Anschauungen bis zu einem gewissen Grade äußerlich gerecht zu werden, mochten dieser Auffassung zu weiterer Verbreitung verhelfen. Und wer sich ihr anschließen wollte, ohne direkt den Assyrer als König anzuerkennen, hielt sich an die hervorragende Stellung und das Wirken seiner babylonischen Gemahlin und betrachtete sie als die eigentliche Herrscherin. So werden die babylonischen Nationalisten strenger Observanz namentlich in der ersten Hälfte von \emph{Adadnirari's} Regierung in der Minderzahl gewesen sein.\footnote{Nach den obigen Darlegungen wird man auch fernerhin nur (S. 264 f. Anm. 5) von einer --- allerdings aus verschiedenen Gründen sehr hohen --- Wahrscheinlichkeit, reden dürfen, dass \emph{Adadnirari} (resp. \emph{Sammuramat}) in der Dynastie H der Königsliste namentlich aufgeführt figurierte.}

\emph{Semiramis} aber die es verstanden hat, Bemühungen um einen Ausgleich der beiden jahrhundertelang verfeindeten Völker zu fördern wenn nicht zu wecken, war sich gewiss im Klaren darüber, dass schließlich der Gewinn dabei den Babyloniern, als den in jeglicher Kultur höherstehenden, zufallen würde. So betrachtet, stellt sich, ganz gegen die Absicht \emph{Adadnirari's}, die Einführung des \emph{Nebo}-Dienstes als eine friedliche babylonische Eroberung dar. In der dabei zutage tretenden diplomatischen Umgehung der Anstöße, der Schonung der nationalen Empfindlichkeit der Assyrer, der scheinbaren Anerkennung ihrer Obmacht (o. S. 267 f.) ist die wirksame Beteiligung der klugen und umsichtigen Frau schwerlich zu verkennen. Der Erfolg hat ihr Recht gegeben: die späteren Assyrerkönige von \emph{Tiglatpileser 3.} (745-27) aus zogen nach Babylon, um das dortige Königtum zum assyrischen hinzuzuerwerben (S. 266). Gewiss hat \emph{Sammuramat}, deren Persönlichkeit, durch den Vergleich mit verwandten Gestalten wie der \emph{Hatšepsowet}\footnote{Vgl. deren Charakteristik und Bildnis, \textsc{Maspero}, \emph{Histoire} 2. p. 239; \textsc{Steindorff}, \emph{Blütezeit des Pharaonenreichs}, S. 33 u. 20.} von Ägypten, der \emph{Arsinoë Philadelphos} und der Kaiserin-Mutter von China an Verständnis gewinnen wird, den bedeutenden Einfluss, den sie auf ihren Gemahl ausgeübt, auch in anderer Richtung geltend gemacht.

Ob sie auch äußerlich die Schranken des Frauengemachs überschritt, etwa ihren Gemahl bei seinen zahlreichen Kriegszügen gelegentlich ins Lager begleitete und durch ihre Gegenwart den Kampfeseifer der Truppen hob, wissen wir nicht. Denkbar ist es sehr wohl,\footnote{Man denke z. B. an Arsinoë, Philopators Schwester, unmittelbar vor der Schlacht von Raphia, Polybios 5. 84.} aber sicher historische Zeugnisse dafür liegen nicht vor.\footnote{Zu Panyassis vgl. unten S. 281. u. Anm. 3.}

Es erübrigt zu zeigen, wie sich aus der historischen \emph{Semiramis} die Sagengestalt entwickelte. Selbstverständlich denke ich nicht daran, hier eine Geschichte der Semiramissage und ihrer Ausbreitung zu geben. Ich erinnere also nur im Vorübergehen daran, dass die spätere Zeit in der Zuweisung von Werken und Taten an die Semiramis weit über Ktesias hinausgegangen ist und dass in diese Kategorie namentlich auch die hängenden Gärten\footnote{Ktesias bei Diodor 2. 10: κρεμαστὸς καλούμενος κῆπος... οὐ Σεμιράμιδος.} gehören. Ebenso betrachte ich es als bekannt und allgemein zugegeben, dass zum Bilde der Romanfigur die babylonisch-assyrische Kriegs- und Liebesgöttin \emph{Ištar} und die sie betreffenden Legenden wesentliche Züge geliefert haben.\footnote{Vgl. \textsc{Maspero}, \emph{Histoire} 1. 580 ff., 2. 618 u. 2. Dadurch dass man dieses sekundäre Element der Sage in den Vordergrund rückt ("`die Semiramis des Ktesias ist eine vermenschlichte, als historische Persönlichkeit und zwar als Herrscherin über Babel und Assur vorgestellte Göttin,"' \textsc{Tiele}, \emph{Geschichte}, S. 213), wird die Entstehung der Sage nicht erklärt.} Es kommt nur darauf an, zu erklären, wieso die Gestalt der Semiramis bis zur Fähigkeit zu solcher Verschmelzung gediehen ist.

\emph{Semiramis} und ihr rein eponymer Gemahl \emph{Ninos} gelten als erste Herrscher Assyriens. Das gibt den entscheidenden Wegweiser. Eine solche Vorstellung kann unmöglich auf assyrischem oder babylonischem Boden erwachsen sein, sondern nur bei einem Fremdvolke. Wenn ein Fremdvolk von primitiven Sitten zur Zeit, da die \emph{Sammuramat} an der Leitung der Geschicke Assyriens beteiligt war, zum erstenmal mit den kriegerischen Assyrern in nähere Berührung kam und von dem Reichtum und der Pracht ihrer Städte hörte, so erklärt es sich vollauf, dass diese Herrscherin als Begründerin der assyrischen Macht und Herrlichkeit betrachtet und zum Mittelpunkt eines Legendenkreises wurde. Unsere Beweiskette ist geschlossen, wenn wir das Volk nachweisen, das zur Zeit \emph{Adadnirari's 3.} und der \emph{Sammuramat} zum erstenmal mit den Assyrern in nachhaltige Feindseligkeiten gerät, und ferner zeigen, dass auf dieses Volk passt, was wir über die Herkunft der Sage wissen oder anderweitig zu vermuten haben.\footnote{Vgl. \emph{Berl. phil. Wochenschr.} 1894, Sp. 239 f.}

Beides trifft zu für die Meder.\footnote{Vgl. \emph{Berl. phil. Wochenschr.} 1894, Sp. 239 f.} Nicht weniger als acht von \emph{Adadnirari's 3.} Regierungsjahren sind nach der "`Verwaltungsliste"' durch Feldzüge gegen die Meder (\emph{Mad-ai(a)}) in Anspruch genommen, und auch in der größeren Palastinschrift des Königs\footnote{I. R. 35 No. 1, Z. 7, KB. 1. 190.} (S. 262) werden diese Meder (\emph{Ma-da-ai(a)}) erwähnt. Und diese Kämpfe bilden die erste ernste und nachhaltige Berührung zwischen beiden Völkern. Von \emph{Adadnirari 3.} ab machen die Meder allen Assyrerkönigen, die überhaupt die Herrschaft im Osten zu sichern oder auszubreiten suchen,\footnote{Namentlich \emph{Tiglatpileser 3.}, \emph{Sargon 2.}, \emph{Assarhaddon}.} schwer zu schaffen. Vor \emph{Adadnirari 3.} werden sie dagegen überhaupt nur ein einziges Mal erwähnt von dessen Vater \emph{Salmanassar 2.}, der in seinem 24. Regierungsjahre (836) unter anderen Völkern auch die \emph{Amadai} bekämpft.\footnote{Vgl. jetzt besonders \textsc{Streck}, ZA. 15. S. 317 ff.; zu \emph{Amadai} = \emph{Madai} speziell ebenda S. 372 und ZA. 14. 139. --- Wer für die Verwertung der Verwaltungsliste auf die Umschrift und Übersetzung in KB. 1. angewiesen ist, sei darauf hingewiesen, dass bekanntlich die Lesung \emph{ana Mad-ai(a)} (Abkürzung für \emph{ana (mât) Mad-ai(a)}), "`gegen die Meder"' (nicht mit KB. 1. \emph{ana (mât)} A. A. "`nach dem Lande A. A."') gesichert ist durch die Schreibung \emph{(mât) Ma-da-aia} an der in Anm. 2 angeführten Stelle bei \emph{Adadnirari 3.}, sowie besonders durch die Variante der Prisma-Inschrift \emph{Sanheribs}: \textsc{Taylor}-Zylinder Col. 2. 50 \emph{(mât) Ma-da-ai(a)}, dafür in dem Duplikat K. 1674: \emph{Mad-ai(a)}, s. KB. 2., 90 Anm. 1.} Also unter dem Vater gleichsam ein erstes Geplänkel mit der Vorhut des eindringenden indogermanischen Volkes, dessen Gros der Sohn zum ersten Male und wiederholt die Spitze zu bieten hat.

Dass aber das, was wir bei Ktesias finden, zum guten Teil als ein, wenn auch durch mancherlei literarische Zutaten ausgeschmückter Niederschlag der medisch-persischen Volkstradition zu betrachten ist --- mag man sie nun als "`Legende,"' "`Gesang,"' "`Novelle,"' "`Mär"'\footnote{\emph{Wochenschr. f. klass. Phil.} 1900, S. 962 Anm. 1.} bezeichnen ---, ist längst vermutet und als wahrscheinlich anerkannt worden.\footnote{\textsc{Duncker}, GA. 2.\textsuperscript{5} 18 f.; \textsc{Noeldeke}, \emph{Aufsätze zur persischen Geschichte}, S. 3 g. E., 14; \textsc{Lehmann}, \emph{Šamaššumukin}, T. 2. S. 106.} Unsere von diesen Erwägungen ganz unabhängige Ermittlung, dass die \emph{Semiramis}-Sage bei den Medern entstanden ist,\footnote{Ein m. E, noch nicht genügend geklärtes Problem liegt vor in dem lebendigen Fortleben der \emph{Semiramis}-Legende bei den Armeniern. Die Stadt \emph{Van} heißt bekanntlich "`\emph{Semiramis}-Stadt,"' \emph{Samiramakert}; der \emph{Mennas-canal} "`\emph{Semiramis}-Fluss;"' auch ein, Überbleibsel aus der ältesten Steinzeit bergender Hügel in der Nähe von Van, ist nach ihr benannt, türkisch \emph{Šamyram-alty}. Wenn man bedenkt, dass der bedeutendste der vorarmenischen Chalderkönige, der in \emph{\d{T}ušpa}-(Van) regierende \emph{Menuas}, Sohn des \emph{Ispuïnis}, (ev. auch schon \emph{Menuas}' Sohn \emph{Argistis 1.}) Zeitgenosse \emph{Adadnirari's 3.} und der \emph{Sammuramat} gewesen ist (die \emph{Naïri}-Züge \emph{Samši-Adad's} [S. 261 Anm. 4] waren gegen \emph{Ispuïnis-Ušpina} gerichtet), und dass ferner zwischen Medern und Armeniern allezeit enge nachbarliche Beziehungen bestanden haben, so könnten Zweifel an den rein literarischen Grundlagen dieser Tradition auftauchen. Doch mag folgendes als Mahnung zur Vorsicht dienen. Der heutige Weg von Bitlis nach Sö'ört führt ca. 4-5 km. unterhalb Bitlis am rechten Ufer des \emph{Bitlis-čai} und hoch über dessen tiefeingeschnittenem Bett durch ein Felsenthor, dass wir im März 1899 passierten. Wenn ich nicht sehr irre, gab uns der Ingenieur des Vilayets Bitlis Herr \textsc{Djoyas} an, dass dieses Felsenthor ein Bestandteil der von ihm begonnenen Straße Bitlis-Sö'ört, also eine moderne Sprengung sei. \textsc{Lynch} aber (\emph{Armenia}, vol. 2. (1901), p. 156) hörte es 1894 als \emph{Semiramis}-Tunnel bezeichnen.} kann nur als eine Bestätigung dafür gelten.

Und wenn nach \textsc{Diels}'\footnote{\emph{SBer. Berl. archäol. Ges.} Nov. 1898 (= \emph{Wochenschr. f. klass. Phil.} 1899, Sp. 27).} überzeugender Konjektur in der Inschrift \emph{IGIns.} 1. 145\footnote{Inschrift (jetzt im Berliner Museum) zu einer Doppelherme des Panyassis und Herodot (\textsc{Winter} bei \textsc{Hiller} v. \textsc{Gaertringen}, \emph{Ath. Mitth.} 1896, 61 f.).} dem Panyassis die Kenntnis der
\begin{quotation}
Ἀσσυρίη[ς ἆθλα] Σεμ[ειρά]μιος
\end{quotation}
\paragraph{}
\hspace*{-3mm}zugeschrieben wird, so stimmt auch diese frühere Erwähnung der sagenhaften\footnote{Bei dem Dichter sowie nach dem Inhalt der auf ihn zu deutenden Worte des Epigramms wird man natürlich zunächst an die sagenhaften Kämpfe der \emph{Semiramis}, nicht an etwaige historische Reminiszenzen (vgl. o. S. 279) zu denken haben.} Kämpfe der \emph{Semiramis} vortrefflich zu unserer Voraussetzung einer volkstümlichen Entstehung und Verbreitung der \emph{Semiramis}-Sage zunächst auf iranischem Boden, die ihrer literarischen Verwertung und Ausgestaltung durch Ktesias vorausgegangen war.\footnote{Nachträge: Zu S. 258 ff. war noch auf \textsc{A. Jeremias}' Artikel \emph{Nebo} in \textsc{Roschers} \emph{Lexikon der Mythologie} 3., Sp. 45-70 zu verweisen. Entgegen der allgemeinen Ansicht, betrachtet er (Sp. 64) die Statuen von Kalach nicht als Bilder des \emph{Nebo}, sondern \emph{Adad-nirari's 3.} Aber für den Gott entscheidet der Hörnerschmuck an der Kopf bedeckung (s. \textsc{Heuzey}, \emph{Les origines orientales de l'art.} 1., 70 ff.). Dieses Merkmal wird noch oft übersehen: auch der auf den Siegelzylindern so häufige Fürsprecher ist ein Gott, kein "`Priester"' schlechthin. Spuren eines Gottkönigtums finden sich zwar im Zweistromland, und ebenso (s. \textsc{G. Hoffmann} ZA 11., 271) "`Göttermaskeraden"' (in diese Richtung wird m. E. für die orientalische Bezeichnung Alexanders des Großen als des "`Hörnerträgers"' mitzusuchen sein). Aber für die Statue eines mit den Attributen der Göttlichkeit bekleideten regierenden Königs gibt es m. W. bisher keine Belege, so viele solche Königsbilder wir haben. Die an die \emph{Gudea}-Statuen erinnernde gefaltete Haltung der Hände, die dann ev. das Fehlen des Schreibgriffels mit sich bringt --- beides fällt \textsc{Jeremias} mit Recht auf --- würde sich gut erklären, wenn \emph{Nebo} unausgesprochenermassen als Fürbitter bei einer höheren Gottheit (oben S. 268, Z. 14 v. u. ff., vgl. allgemein \textsc{Zimmern}, \emph{Vater, Sohn und Fürsprecher}) dargestellt wäre, wobei sich dann Assyrer und Babylonier (S. 267 f. u. 277 f.) jede ihr Teil denken konnten. --- In Heft 1 von \textsc{Belcks} "`\emph{Beiträgen zur Geographie und Geschichte Vorderasiens},"' das erschien, nachdem Bogen 17 gesetzt war, finden sich Erörterungen auch über die babylonische Dynastie H. (o. S. 264 Anm. 5), die synchronistische Geschichte und \emph{Adadnirari 3.}, darunter solche, die sich teils mit Obigem berühren (so die Richtigstellung der Züge \emph{Šamši-Adad's} [oben S. 261 f. Anm. 4]), teils dadurch widerlegt werden (so die auf der falschen Ergänzung des Namens \emph{Bau-a\U{h}-iddin} in KB. 1. [s. oben gleichenorts] gegründete Behauptung einer gleichzeitigen Regierung \emph{Marduk-balâ(ṭ)-su-i\d{k}bî's} und \emph{Bau-a\U{h}-iddin's}).}
\clearpage
\end{document}
